\documentclass[14pt,compress,aspectratio=169]{beamer}


% Modified from: generic-ornate-15min-45min.de.tex
\mode<presentation>
{
  \usetheme{Warsaw}
  \useoutertheme{infolines}
  \setbeamercovered{invisible}
}

\usepackage[english]{babel}
\usepackage[latin1]{inputenc}
%\usepackage{times}
\usepackage[T1]{fontenc}

% Taken from Fernando's slides.
\usepackage{ae,aecompl}
\usepackage{mathpazo,courier,euler}
\usepackage[scaled=.95]{helvet}

\definecolor{darkgreen}{rgb}{0,0.5,0}

\usepackage{listings}
\lstset{language=Python,
    basicstyle=\ttfamily\bfseries,
    commentstyle=\color{red}\itshape,
  stringstyle=\color{darkgreen},
  showstringspaces=false,
  keywordstyle=\color{blue}\bfseries}

%%%%%%%%%%%%%%%%%%%%%%%%%%%%%%%%%%%%%%%%%%%%%%%%%%%%%%%%%%%%%%%%%%%%%%
% Macros
\setbeamercolor{emphbar}{bg=blue!20, fg=black}
\newcommand{\emphbar}[1]
{\begin{beamercolorbox}[rounded=true]{emphbar}
      {#1}
 \end{beamercolorbox}
}
\newcounter{time}
\setcounter{time}{0}
\newcommand{\inctime}[1]{\addtocounter{time}{#1}{\tiny \thetime\ m}}

\newcommand{\typ}[1]{\textbf{\texttt{{#1}}}}


\newcommand{\kwrd}[1]{ \texttt{\textbf{\color{blue}{#1}}}  }

%%% This is from Fernando's setup.
% \usepackage{color}
% \definecolor{orange}{cmyk}{0,0.4,0.8,0.2}
% % Use and configure listings package for nicely formatted code
% \usepackage{listings}
% \lstset{
%    language=Python,
%    basicstyle=\small\ttfamily,
%    commentstyle=\ttfamily\color{blue},
%    stringstyle=\ttfamily\color{orange},
%    showstringspaces=false,
%    breaklines=true,
%    postbreak = \space\dots
% }

%\pgfdeclareimage[height=0.75cm]{iitmlogo}{iitmlogo}
%\logo{\pgfuseimage{iitmlogo}}


%% Delete this, if you do not want the table of contents to pop up at
%% the beginning of each subsection:
\AtBeginSubsection[]
{
  \begin{frame}<beamer>
    \frametitle{Outline}
    \tableofcontents[currentsection,currentsubsection]
  \end{frame}
}

\AtBeginSection[]
{
  \begin{frame}<beamer>
    \frametitle{Outline}
    \tableofcontents[currentsection,currentsubsection]
  \end{frame}
}

% If you wish to uncover everything in a step-wise fashion, uncomment
% the following command:
%\beamerdefaultoverlayspecification{<+->}

%\includeonlyframes{current,current1,current2,current3,current4,current5,current6}


\title[Practice advanced functions]{Advanced Python}
\subtitle{Practice advanced functions}

\author[FOSSEE] {The FOSSEE Group}

\institute[IIT Bombay] {Department of Aerospace Engineering\\IIT Bombay}
\date[] {Mumbai, India}

\begin{document}

\begin{frame}
  \titlepage
\end{frame}

\begin{frame}[plain, fragile]
  \frametitle{Exercise: product of all arguments}
  \begin{block}{}
    Write a function called \lstinline{prod} that can be called with any
    number of arguments (at least one) but computes the product of all the
    given values as seen below.
  \end{block}

\begin{lstlisting}
In []: prod(2.,  2.)
Out[]: 4.0

In []: prod(1, 2, 3)
Out[]: 6
\end{lstlisting}
\end{frame}

\begin{frame}[plain, fragile]
  \frametitle{Solution}
\begin{lstlisting}
def prod(*args):
    res = 1.0
    for x in args:
        res *= x
    return res

\end{lstlisting}
\end{frame}

\begin{frame}[plain, fragile]
  \frametitle{Another solution}
\begin{lstlisting}
def prod(*args):
    res = args[0]
    for i in range(1, len(args)):
        res *= args[i]
    return res

\end{lstlisting}
\end{frame}


\begin{frame}[plain, fragile]
  \frametitle{Exercise: function applied to arguments}
  \begin{block}{}
    Write a function called \lstinline{apply} which takes a function followed
    by an arbitrary number of arguments and returns a list with the function
    applied to all the arguments as shown below.
  \end{block}

\begin{lstlisting}
In []: def twice(x):
  ...:     return x*2

In []: apply(twice,  1, 2)
Out[]: [2, 4]

In []: apply(twice,  1, 2, 3)
Out[]: [2, 4, 6]

\end{lstlisting}
\end{frame}


\begin{frame}[plain, fragile]
  \frametitle{Solution}
\begin{lstlisting}
def apply(f, *args):
    result = []
    for arg in args:
        result.append(f(arg))
    return result
\end{lstlisting}
\end{frame}

\begin{frame}[plain, fragile]
  \frametitle{Exercise: number of kwargs}
  \begin{block}{}
    Write a function \lstinline{nkw} that takes an arbitrary number of
    keyword arguments and returns the number of keyword arguments passed.
  \end{block}

\begin{lstlisting}
In []: nkw(x=1, y=2)
Out[]: 2

In []: nkw()
Out[]: 0

In []: nkw(x=1)
Out[]: 1
\end{lstlisting}
\end{frame}


\begin{frame}[plain, fragile]
  \frametitle{Solution}
\begin{lstlisting}
def nkw(**kw):
    return len(kw)
\end{lstlisting}
\end{frame}

\begin{frame}[plain, fragile]
  \frametitle{Exercise: name of kwargs}
  \begin{block}{}
    Write a function \lstinline{kwname} that takes an arbitrary number of
    keyword arguments and returns a sorted list of the keyword arguments
    passed.
  \end{block}

\begin{lstlisting}
In []: kwname(x=1, y=2)
Out[]: ['x', 'y']

In []: kwname()
Out[]: []

In []: kwname(z=1, a=2)
Out[]: ['a', 'z']
\end{lstlisting}
\end{frame}


\begin{frame}[plain, fragile]
  \frametitle{Solution}
\begin{lstlisting}
def kwname(**kw):
    return sorted(kw.keys())
\end{lstlisting}
\end{frame}

\begin{frame}[plain, fragile]
  \frametitle{Exercise: power function}
  \begin{block}{}
    Write a function called \lstinline{power} that is given a single integer
    and returns a function that takes a single number but returns its power.
    For example:
  \end{block}

\begin{lstlisting}
In []: pow2 = power(2)
In []: pow2(2)
Out[]: 4
In []: pow2(4)
Out[]: 16

In []: pow3 = power(3)
In []: pow3(2)
Out[]: 8
\end{lstlisting}
  Hint: this is a closure.
\end{frame}


\begin{frame}[plain, fragile]
  \frametitle{Solution}
\begin{lstlisting}
def power(n):
    def g(x):
        return x**n
    return g

\end{lstlisting}
\end{frame}


\begin{frame}[plain, fragile]
  \frametitle{Exercise: debug function}
  \begin{block}{}
    Write a function called \lstinline{debug} that takes any function as a
    single positional argument but returns a function that first prints out
    the arguments passed to the function before calling it and returning its
    value. For example:
  \end{block}

\begin{lstlisting}
In []: debug(max)(1, 2)
(1, 2) {}
Out[]: 2

In []: import math

In []: debug(math.sin)(1.0)
(1.0,) {}
Out[]: 0.8414709848078965

\end{lstlisting}
\end{frame}


\begin{frame}[plain, fragile]
  \frametitle{Solution}
\begin{lstlisting}
def debug(f):
    def g(*args, **kw):
        print(args, kw)
        return f(*args, **kw)
    return g
\end{lstlisting}
\end{frame}


\end{document}
