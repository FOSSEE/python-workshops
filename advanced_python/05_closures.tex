\documentclass[14pt,compress,aspectratio=169]{beamer}


% Modified from: generic-ornate-15min-45min.de.tex
\mode<presentation>
{
  \usetheme{Warsaw}
  \useoutertheme{infolines}
  \setbeamercovered{invisible}
}

\usepackage[english]{babel}
\usepackage[latin1]{inputenc}
%\usepackage{times}
\usepackage[T1]{fontenc}

% Taken from Fernando's slides.
\usepackage{ae,aecompl}
\usepackage{mathpazo,courier,euler}
\usepackage[scaled=.95]{helvet}

\definecolor{darkgreen}{rgb}{0,0.5,0}

\usepackage{listings}
\lstset{language=Python,
    basicstyle=\ttfamily\bfseries,
    commentstyle=\color{red}\itshape,
  stringstyle=\color{darkgreen},
  showstringspaces=false,
  keywordstyle=\color{blue}\bfseries}

%%%%%%%%%%%%%%%%%%%%%%%%%%%%%%%%%%%%%%%%%%%%%%%%%%%%%%%%%%%%%%%%%%%%%%
% Macros
\setbeamercolor{emphbar}{bg=blue!20, fg=black}
\newcommand{\emphbar}[1]
{\begin{beamercolorbox}[rounded=true]{emphbar}
      {#1}
 \end{beamercolorbox}
}
\newcounter{time}
\setcounter{time}{0}
\newcommand{\inctime}[1]{\addtocounter{time}{#1}{\tiny \thetime\ m}}

\newcommand{\typ}[1]{\textbf{\texttt{{#1}}}}


\newcommand{\kwrd}[1]{ \texttt{\textbf{\color{blue}{#1}}}  }

%%% This is from Fernando's setup.
% \usepackage{color}
% \definecolor{orange}{cmyk}{0,0.4,0.8,0.2}
% % Use and configure listings package for nicely formatted code
% \usepackage{listings}
% \lstset{
%    language=Python,
%    basicstyle=\small\ttfamily,
%    commentstyle=\ttfamily\color{blue},
%    stringstyle=\ttfamily\color{orange},
%    showstringspaces=false,
%    breaklines=true,
%    postbreak = \space\dots
% }

%\pgfdeclareimage[height=0.75cm]{iitmlogo}{iitmlogo}
%\logo{\pgfuseimage{iitmlogo}}


%% Delete this, if you do not want the table of contents to pop up at
%% the beginning of each subsection:
\AtBeginSubsection[]
{
  \begin{frame}<beamer>
    \frametitle{Outline}
    \tableofcontents[currentsection,currentsubsection]
  \end{frame}
}

\AtBeginSection[]
{
  \begin{frame}<beamer>
    \frametitle{Outline}
    \tableofcontents[currentsection,currentsubsection]
  \end{frame}
}

% If you wish to uncover everything in a step-wise fashion, uncomment
% the following command:
%\beamerdefaultoverlayspecification{<+->}

%\includeonlyframes{current,current1,current2,current3,current4,current5,current6}


\title[Closures]{Advanced Python}
\subtitle{Closures}

\author[FOSSEE] {The FOSSEE Group}

\institute[IIT Bombay] {Department of Aerospace Engineering\\IIT Bombay}
\date[] {Mumbai, India}

\begin{document}

\begin{frame}
  \titlepage
\end{frame}

\begin{frame}
  \frametitle{Overview}
  \begin{itemize}
  \item Higher-order functions
  \item Closures
  \end{itemize}
\end{frame}

\begin{frame}[fragile]
  \frametitle{Recap of higher-order functions}
  \begin{lstlisting}
In []: def sqr(x):
  ...:    return x*x

In []: def sum(func, n):
  ...:     result = 0
  ...:     for i in range(n):
  ...:         result += func(i)
  ...:     return result

In []: sum(sqr, 5)
Out[]: 30
  \end{lstlisting}
\end{frame}

\begin{frame}[fragile]
  \frametitle{Recap of higher-order functions}
  \begin{lstlisting}
def f():
    def g(x):
        return x+1
    return g

In []: func = f()
In []: func(1)
In []: f()(1)  # Also valid!
  \end{lstlisting}
\end{frame}


\begin{frame}[fragile]
  \frametitle{Closures}
  \begin{lstlisting}
def mul(x):
    def g(y):
        return y*x
    return g

In []: twice = mul(2.0)
In []: twice(20)
Out[]: 40.0
  \end{lstlisting}
\end{frame}

\begin{frame}[fragile]
  \frametitle{Closures ...}
  \begin{lstlisting}

In []: thrice = mul(3.0)
In []: thrice(20)
Out[]: 60.0
  \end{lstlisting}
\end{frame}

\begin{frame}
  \frametitle{Discussion}
  \begin{itemize}
  \item \typ{g} uses \typ{x}
  \item \typ{x} is different in each case
  \item So \typ{twice, thrice} ``curry'' the passed \typ{x}
  \item This function is called a closure
  \item Encloses its local environment
  \end{itemize}
\end{frame}


\begin{frame}[fragile]
  \frametitle{Summary: Higher-order functions}
  \begin{itemize}
  \item Functions that manipulate functions
  \item Passing a function as an argument
  \item A function that returns another function
  \item A closure
  \end{itemize}
\end{frame}

\end{document}
