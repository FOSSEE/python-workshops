\documentclass[14pt,compress,aspectratio=169]{beamer}

\usepackage{hyperref}

% Modified from: generic-ornate-15min-45min.de.tex
\mode<presentation>
{
  \usetheme{Madrid}
  \useoutertheme{infolines}
  \setbeamercovered{invisible}
}

\usepackage[english]{babel}
\usepackage[latin1]{inputenc}
%\usepackage{times}
\usepackage[T1]{fontenc}

% Taken from Fernando's slides.
\usepackage{ae,aecompl}
\usepackage{mathpazo,courier,euler}
\usepackage[scaled=.95]{helvet}

\definecolor{darkgreen}{rgb}{0,0.5,0}

\usepackage{listings}
\lstset{language=Python,
    basicstyle=\ttfamily\bfseries,
    commentstyle=\color{red}\itshape,
  stringstyle=\color{darkgreen},
  showstringspaces=false,
  keywordstyle=\color{blue}\bfseries}

%%%%%%%%%%%%%%%%%%%%%%%%%%%%%%%%%%%%%%%%%%%%%%%%%%%%%%%%%%%%%%%%%%%%%%
% Macros
\setbeamercolor{emphbar}{bg=blue!20, fg=black}
\newcommand{\emphbar}[1]
{\begin{beamercolorbox}[rounded=true]{emphbar}
      {#1}
 \end{beamercolorbox}
}
\newcounter{time}
\setcounter{time}{0}
\newcommand{\inctime}[1]{\addtocounter{time}{#1}{\tiny \thetime\ m}}

\newcommand{\typ}[1]{\textbf{\texttt{{#1}}}}


\newcommand{\kwrd}[1]{ \texttt{\textbf{\color{blue}{#1}}}  }

%%% This is from Fernando's setup.
% \usepackage{color}
% \definecolor{orange}{cmyk}{0,0.4,0.8,0.2}
% % Use and configure listings package for nicely formatted code
% \usepackage{listings}
% \lstset{
%    language=Python,
%    basicstyle=\small\ttfamily,
%    commentstyle=\ttfamily\color{blue},
%    stringstyle=\ttfamily\color{orange},
%    showstringspaces=false,
%    breaklines=true,
%    postbreak = \space\dots
% }

%\pgfdeclareimage[height=0.75cm]{iitmlogo}{iitmlogo}
%\logo{\pgfuseimage{iitmlogo}}


%% Delete this, if you do not want the table of contents to pop up at
%% the beginning of each subsection:
\AtBeginSubsection[]
{
  \begin{frame}<beamer>
    \frametitle{Outline}
    \tableofcontents[currentsection,currentsubsection]
  \end{frame}
}

\AtBeginSection[]
{
  \begin{frame}<beamer>
    \frametitle{Outline}
    \tableofcontents[currentsection,currentsubsection]
  \end{frame}
}

% If you wish to uncover everything in a step-wise fashion, uncomment
% the following command:
%\beamerdefaultoverlayspecification{<+->}

%\includeonlyframes{current,current1,current2,current3,current4,current5,current6}



\title[OOP: more special methods]{Advanced Python}
\subtitle{Object Oriented Programming: more special methods}

\author[FOSSEE] {The FOSSEE Group}

\institute[IIT Bombay] {Department of Aerospace Engineering\\IIT Bombay}
\date[] {Mumbai, India}

\begin{document}

\begin{frame}
  \titlepage
\end{frame}

\begin{frame}
  \frametitle{Special methods}
  \begin{itemize}
  \item \py{__init__} is one such method
  \item \py{__str__, __repr__}
  \item Many more
  \end{itemize}
\end{frame}

\begin{frame}[fragile]
  \frametitle{Motivation}
\begin{lstlisting}
class A:
    def __init__(self, x=0):
        self.x = x

In []: a, b = A(), A()
\end{lstlisting}
\pause
\begin{lstlisting}
In []: a == b
Out[]: False

In []: a < b # ?
\end{lstlisting}
\end{frame}


\begin{frame}[fragile]
  \frametitle{Comparison operators: rich comparison}
  \begin{itemize}
  \item \py{__lt__(self, other)}: \py{self < other}
  \item \py{__le__(self, other)}: \py{self <= other}
  \item \py{__gt__(self, other)}: \py{self > other}
  \item \py{__ge__(self, other)}: \py{self >= other}
  \item \py{__eq__(self, other)}: \py{self == other}
  \item \py{__ne__(self, other)}: \py{self != other}
    \vspace*{0.2in}
  \item \py{__bool__(self)}: used for truth values
  \end{itemize}
\end{frame}

\begin{frame}[fragile]
  \frametitle{Example}
  \small
\begin{lstlisting}
class A:
    def __init__(self, x=0):
        self.x = x
    def __eq__(self, other):
        return self.x == other.x
    def __lt__(self, other):
        return self.x < other.x
\end{lstlisting}
  \pause
\begin{lstlisting}
In []: a, b = A(), B()
In []: a == b
Out[]: True

In []: a.x = 1
In []: a == b
Out[]: False
\end{lstlisting}
\end{frame}


\begin{frame}[fragile]
  \frametitle{Container types}
  \begin{itemize}
  \item \py{__len__(self)}: \py{len(object)}
  \item \py{__getitem__(self, key)}: \py{object[key]}
  \item \py{__setitem__(self, key, value)}: \py{object[key] = value}
  \item \py{__delitem__(self, name)}: \py{del object[key]}
  \item \py{__iter__(self)}: \py{for x in object}
  \item \py{__contains__(self, item)}: \py{item in object}

  \end{itemize}
\end{frame}

\begin{frame}
  \frametitle{Emulating numeric types}
  \begin{itemize}
  \item \py{__add__(self, other)}: \py{object + other}
  \item \py{__sub__(self, other)}: \py{object - other}
  \item \py{__mul__(self, other)}: \py{object * other}
  \item \py{__div__(self, other)}: \py{object / other}
  \item \py{__matmul__(self, other)}: \py{object @ other}
  \item ...

  \item \py{__radd__(self, other)}: \py{other + object}
  \item \py{__rsub__(self, other)}: \py{other - object}
  \item More similar
  \end{itemize}
\end{frame}

\begin{frame}
  \frametitle{Emulating numeric types}
  \begin{itemize}
  \item \py{__iadd__(self, other)}: \py{object += other}
  \item \py{__isub__(self, other)}: \py{object -= other}
  \item ...

  \item \py{__neg__(self)}: \py{-object}
  \item \py{__pos__(self)}: \py{+object}
  \item \py{__abs__(self)}: \py{abs(object)}
  \item  ...

  \item \py{__int__(self)}: \py{int(object)}
  \item \py{__float__(self)}: \py{float(object)}

  \end{itemize}
\end{frame}


\begin{frame}[fragile]
  \frametitle{Example}
\begin{lstlisting}
class A:
    def __init__(self, x=0):
        self.x = x
    def __add__(self, other):
        return A(self.x + other.x)

In []: a, b = A(1), A(1)
In []: c = a + b
In []: print(c.x)
2
\end{lstlisting}
\end{frame}


\begin{frame}[fragile]
  \frametitle{Callable objects}
  \begin{itemize}
  \item \py{__call__(self [, args...])}: \py{object()}
  \end{itemize}
\end{frame}

\begin{frame}[fragile]
  \frametitle{Attribute access}
  \begin{itemize}
  \item \py{__getattr__(self, name)}: \py{self.name}
  \item \py{__getattribute__(self, name)}: \py{self.name}
  \item \py{__setattr__(self, name, value)}: \py{self.name = value}
  \item \py{__delattr__(self, name)}: \py{del self.name}
  \item \py{__dir__(self)}: \py{dir(object)}
  \end{itemize}
\end{frame}


\begin{frame}
  \frametitle{Doing more}
  \begin{itemize}
  \item Explore later
  \item See: \url{docs.python.org/reference/datamodel.html}
  \end{itemize}

\end{frame}

\begin{frame}[plain, fragile]
  \frametitle{Exercise: Animal weights}
  \begin{block}{}
    Create an \py{Animal} class with a \py{name} (str) and \py{weight} (float)
    attribute.  Allow a user to perform rich comparisons of animals based on
    their weight (it should support all comparisons). For example:
  \end{block}

\begin{lstlisting}
In []: cat = Animal('cat', 5)
In []: dog = Animal('dog', 10)
In []: dog > cat
Out[]: True
\end{lstlisting}
\end{frame}


\begin{frame}[plain, fragile]
  \frametitle{Solution}
  \small
\begin{lstlisting}
class Animal:
    def __init__(self, name, weight=0.0):
        self.name = name
        self.weight = weight
    def __lt__(self, other):
        return self.weight < other.weight
    def __le__(self, other):
        return self.weight <= other.weight
    def __eq__(self, other):
        return self.weight == other.weight
    # ...
\end{lstlisting}
\end{frame}

\begin{frame}[plain, fragile]
  \frametitle{Exercise: Zoo class container}
  \begin{block}{}
    Create a \py{Zoo} class which can contain animals.  The Zoo should behave
    like a container.
  \end{block}

\begin{lstlisting}
In []: a = Animal('cat', 5)
In []: b = Animal('dog', 10)
In []: zoo = Zoo(a, b)
In []: len(zoo)
Out[]: 2
In []: a in zoo
Out[]: True
\end{lstlisting}
  Implement all other container related methods.
\end{frame}


\begin{frame}[plain, fragile]
  \frametitle{Solution}
  \vspace*{-0.1in}
  \small
\begin{lstlisting}
class Animal:
    def __init__(self, name, weight=0.0):
        self.name = name
        self.weight = weight

class Zoo:
    def __init__(self, *animals):
        self.animals = list(animals)
    def __len__(self):
        return len(self.animals)
    def __contains__(self, item):
        return item in self.animals
    def __getitem__(self, key):
        return self.animals[key]
    def __iter__(self):
        return self.animals
\end{lstlisting}
\end{frame}



\begin{frame}[plain, fragile]
  \frametitle{Exercise: \py{Point} class addition}
  \begin{block}{}
    Write a \py{Point} class with two instance attributes \py{x, y}.  Make it
    possible to add and subtract two points.  Also allow \py{abs()} to be
    callable on a point.
  \end{block}

\begin{lstlisting}
In []: p1 = Point(1, 2)
In []: p2 = Point(1, 0)
In []: p = p1 + p2
In []: p.x, p.y
Out[]: (2, 2)
In []: abs(p)
Out[]: 2.8284271247461903
\end{lstlisting}
\end{frame}


\begin{frame}[plain, fragile]
  \frametitle{Solution}
  \small
\begin{lstlisting}
from math import sqrt

class Point:
    def __init__(self, x=0, y=0):
        self.x = x
        self.y = y
    def __add__(self, other):
        return Point(self.x + other.x, self.y + other.y)
    def __sub__(self, other):
        return Point(self.x - other.x, self.y - other.y)
    def __abs__(self):
        return sqrt(self.x*self.x + self.y*self.y)

\end{lstlisting}
\end{frame}


\end{document}
