\documentclass[14pt,compress,aspectratio=169]{beamer}

\usepackage{hyperref}

% Modified from: generic-ornate-15min-45min.de.tex
\mode<presentation>
{
  \usetheme{Madrid}
  \useoutertheme{infolines}
  \setbeamercovered{invisible}
}

\usepackage[english]{babel}
\usepackage[latin1]{inputenc}
%\usepackage{times}
\usepackage[T1]{fontenc}

% Taken from Fernando's slides.
\usepackage{ae,aecompl}
\usepackage{mathpazo,courier,euler}
\usepackage[scaled=.95]{helvet}

\definecolor{darkgreen}{rgb}{0,0.5,0}

\usepackage{listings}
\lstset{language=Python,
    basicstyle=\ttfamily\bfseries,
    commentstyle=\color{red}\itshape,
  stringstyle=\color{darkgreen},
  showstringspaces=false,
  keywordstyle=\color{blue}\bfseries}

%%%%%%%%%%%%%%%%%%%%%%%%%%%%%%%%%%%%%%%%%%%%%%%%%%%%%%%%%%%%%%%%%%%%%%
% Macros
\setbeamercolor{emphbar}{bg=blue!20, fg=black}
\newcommand{\emphbar}[1]
{\begin{beamercolorbox}[rounded=true]{emphbar}
      {#1}
 \end{beamercolorbox}
}
\newcounter{time}
\setcounter{time}{0}
\newcommand{\inctime}[1]{\addtocounter{time}{#1}{\tiny \thetime\ m}}

\newcommand{\typ}[1]{\textbf{\texttt{{#1}}}}


\newcommand{\kwrd}[1]{ \texttt{\textbf{\color{blue}{#1}}}  }

%%% This is from Fernando's setup.
% \usepackage{color}
% \definecolor{orange}{cmyk}{0,0.4,0.8,0.2}
% % Use and configure listings package for nicely formatted code
% \usepackage{listings}
% \lstset{
%    language=Python,
%    basicstyle=\small\ttfamily,
%    commentstyle=\ttfamily\color{blue},
%    stringstyle=\ttfamily\color{orange},
%    showstringspaces=false,
%    breaklines=true,
%    postbreak = \space\dots
% }

%\pgfdeclareimage[height=0.75cm]{iitmlogo}{iitmlogo}
%\logo{\pgfuseimage{iitmlogo}}


%% Delete this, if you do not want the table of contents to pop up at
%% the beginning of each subsection:
\AtBeginSubsection[]
{
  \begin{frame}<beamer>
    \frametitle{Outline}
    \tableofcontents[currentsection,currentsubsection]
  \end{frame}
}

\AtBeginSection[]
{
  \begin{frame}<beamer>
    \frametitle{Outline}
    \tableofcontents[currentsection,currentsubsection]
  \end{frame}
}

% If you wish to uncover everything in a step-wise fashion, uncomment
% the following command:
%\beamerdefaultoverlayspecification{<+->}

%\includeonlyframes{current,current1,current2,current3,current4,current5,current6}



\title[Decorators]{Advanced Python}
\subtitle{Decorators}

\author[FOSSEE] {The FOSSEE Group}

\institute[IIT Bombay] {Department of Aerospace Engineering\\IIT Bombay}
\date[] {Mumbai, India}

\begin{document}

\begin{frame}
  \titlepage
\end{frame}

\begin{frame}[fragile]
  \frametitle{Overview}
  Decorators:
  \begin{itemize}
  \item transform a function/method using a function
  \item are higher order functions
  \item provide a handy syntax
  \item are a powerful feature
  \item used by many libraries
  \end{itemize}
\end{frame}

\begin{frame}[fragile]
  \frametitle{Trivial example}
\begin{lstlisting}
def deco(func):
    return func

@deco
def greet():
    print("Namaste!")

# @deco is equivalent to:
greet = deco(greet)
\end{lstlisting}
\end{frame}

\begin{frame}[fragile]
  \frametitle{Understanding a bit more}
\begin{lstlisting}
def deco(func):
    print("deco")  # <--
    return func

@deco
def greet():
    print("Namaste!")

greet = deco(greet)
\end{lstlisting}
\end{frame}

\begin{frame}
  \frametitle{Observations}
  \begin{itemize}
  \item Decorator is a convenient syntax
  \item Called when the function is defined
  \item Not called every time the decorated function is called
  \end{itemize}
\end{frame}

\begin{frame}[fragile, plain]
  \frametitle{Non-trivial example}
Modify the function to print ``Hello'' before the function is called
\begin{lstlisting}
def deco(func):
    def new_func(*args, **kw):
        print("Hello")
        return func(*args, **kw)
    return new_func
\end{lstlisting}
\pause
\begin{lstlisting}
@deco
def greet():
    '''Print greeting.'''
    print("Namaste!")

In []: greet()

\end{lstlisting}
\end{frame}

\begin{frame}
  \frametitle{Observations}
  \begin{itemize}
  \item Notice the use of the \py{*args, **kw}
  \item The target function may have any arguments
  \item We replace the original function with \py{new_func}
  \item So does this cause any problems?
  \end{itemize}
\end{frame}

\begin{frame}[fragile]
  \frametitle{Issues}
  \begin{lstlisting}
    In []: greet.__name__
    Out[]: 'new_func'

    In []: greet?

    In []: greet.__doc__

    In []: greet.__module__
  \end{lstlisting}
\end{frame}

\begin{frame}[fragile]
  \frametitle{Using \py{wraps}}
  \begin{itemize}
  \item Using \py{functools.wraps} prevents these problems
  \end{itemize}

  \begin{lstlisting}
from functools import wraps  # <--

def deco(func):
    @wraps(func)  # <--
    def new_func(*args, **kw):
        print("Hello")
        return func(*args, **kw)
    return new_func
\end{lstlisting}

Now try again.
\end{frame}


\begin{frame}[fragile]
  \frametitle{Decorators taking arguments}
  \begin{itemize}
  \item Sometimes you want to customize the decorator
  \item You wish to pass arguments to the decorator
  \end{itemize}
\pause
First attempt:
  \begin{lstlisting}
from functools import wraps

def deco(func, greet='Hello'):
    @wraps(func)
    def new_func(*args, **kw):
        print(greet)
        return func(*args, **kw)
    return new_func
\end{lstlisting}
\end{frame}

\begin{frame}[fragile]
  \frametitle{Let us try}
  \small
  \begin{lstlisting}
@deco
def f():
    print('Hi')

In []: f()
Hello
Hi
\end{lstlisting}
\pause
\begin{lstlisting}
@deco(greet='Namaste')
def f():
    print('Hi')

TypeError: deco() got an unexpected keyword argument 'greet'
  \end{lstlisting}
\end{frame}

\begin{frame}
  \frametitle{Fixing the problem}
  \begin{itemize}
  \item \py{deco(greet='Namaste')} is not passed the function!
  \item So \py{deco(greet='Namaste')} should return a decorator
    \vspace*{0.2in}
  \end{itemize}
\pause
  \begin{block}{Solution}
    \begin{itemize}
    \item Must return a decorator when called without a function
    \item When passed a function just call the decorator
    \end{itemize}
  \end{block}
\end{frame}

\begin{frame}[fragile, plain]
  \frametitle{Fixing the problems}
  \small
  \begin{lstlisting}
def deco(func=None, greet='Hello'):
    def wrapper(func):  # <-- a decorator
        @wraps(func)
        def new_func(*args, **kw):
            print(greet)
            return func(*args, **kw)
        return new_func

    if func is None:
        return wrapper
    else:
        return wrapper(func)

  \end{lstlisting}
\end{frame}

\begin{frame}[fragile]
  \frametitle{Let us try}
  \small
  \begin{lstlisting}
@deco
def f():
    print('Hi')

In []: f()
Hello
Hi
\end{lstlisting}
\pause
\begin{lstlisting}
@deco(greet='Namaste')
def f():
    print('Hi')

In []: f()
Namaste
Hi

  \end{lstlisting}
\end{frame}

\begin{frame}
  \frametitle{Summary}
  \begin{itemize}
  \item Creating simple decorators
  \item Using \py{functools.wraps}
  \item Decorators taking arguments
  \end{itemize}
\end{frame}

\begin{frame}[plain, fragile]
  \frametitle{Exercise: simplest decorator}
  \begin{block}{}
    Write a decorator called \py{greet} that prints \py{'Hello'} before
    the function is called.
  \end{block}
  \begin{lstlisting}
    @greet
    def f(x):
        print(x)

    In []: f(1)
    Hello
    1
\end{lstlisting}
\end{frame}


\begin{frame}[plain, fragile]
  \frametitle{Solution}
\begin{lstlisting}
def greet(func):
    def new_func(*args, **kw):
        print("Hello")
        return func(*args, **kw)
    return new_func
\end{lstlisting}
\end{frame}

\begin{frame}[plain, fragile]
  \frametitle{Exercise: add goodbye}
  \begin{block}{}
    Modify decorator \py{greet} to print ``goodbye'' after executing the
    function.
  \end{block}
  \begin{lstlisting}
from functools import wraps
def greet(func):
    @wraps(func)
    def new_func(*args, **kw):
        print("Hello")
        return func(*args, **kw)
    return new_func
\end{lstlisting}

\end{frame}

\begin{frame}[fragile]
  \frametitle{Solution}
\begin{lstlisting}
def greet(func):
    @wraps(func)
    def new_func(*args, **kw):
        print("Hello")
        result = func(*args, **kw)
        print("goodbye")
        return result
    return new_func
\end{lstlisting}
\end{frame}

\begin{frame}[plain, fragile]
  \frametitle{Exercise: print function name}
  \begin{block}{}
    Write a decorator called \py{debug()} that prints the name of the function
    before it is called.  \textbf{Hint:} Recall \py{__name__}
  \end{block}

\begin{lstlisting}
    @debug
    def my_func(x):
        print(x)

    In []: my_func(1)
    my_func
    1
\end{lstlisting}
\end{frame}


\begin{frame}[plain, fragile]
  \frametitle{Solution}
\begin{lstlisting}
from functools import wraps

def debug(func):
    @wraps(func)
    def new_func(*args, **kw):
        print(func.__name__)
        return func(*args, **kw)
    return new_func
\end{lstlisting}
\end{frame}


\begin{frame}[plain, fragile]
  \frametitle{Exercise: print function name with optional message}
  \begin{block}{}
    Write a decorator called \py{debug(f, message='')} that prints the name of
    the function before it is called along with an optional message.
  \end{block}

\begin{lstlisting}
    @debug(message='DEBUG: ')
    def my_func(x):
        print(x)

    In []: my_func(1)
    DEBUG: my_func
    1
\end{lstlisting}
\end{frame}


\begin{frame}[plain, fragile]
  \frametitle{Solution}
  \vspace*{-0.1in}
  \small
\begin{lstlisting}
from functools import wraps

def debug(func=None, message=''):
    def wrapper(func)
        @wraps(func)
        def new_func(*args, **kw):
            if message:
                print(message, func.__name__)
            else:
                print(func.__name__)
            return func(*args, **kw)
        return new_func
    if func:
        return wrapper(func)
    else:
        return wrapper
\end{lstlisting}
\end{frame}


\begin{frame}[plain, fragile]
  \frametitle{Exercise: track number of calls}
  \begin{block}{}
    Write a decorator called \py{counter()} that keeps track of the number of
    times a decorated function is called.  Write another function called
    \py{show_counts()} which lists all the called functions.
  \end{block}

  \small
\begin{lstlisting}
@counter
def sq(x): return x*x

@counter
def g(x): return x+2

In []: for i in range(10): g(sq(i))
In []: show_counts()
sq : 10
g : 10

\end{lstlisting}
\end{frame}


\begin{frame}[plain, fragile]
  \frametitle{Solution}
\begin{lstlisting}
_counter_data = {}

def counter(func):
    @wraps(func)
    def new_func(*args, **kw):
        if func in _counter_data:
            _counter_data[func] += 1
        else:
            _counter_data[func] = 1
        return func(*args, **kw)
    return new_func
\end{lstlisting}
\end{frame}

\begin{frame}[plain, fragile]
  \frametitle{Solution}
\begin{lstlisting}
def show_counts():
    for f, c in _counter_data.items():
        print(f.__name__, ':', c)
\end{lstlisting}
\end{frame}

\begin{frame}
  \frametitle{Homework}
  \begin{block}{}
    Convert the above into a class so that the global \py{_counter_data} and
    the decorator is encapsulated into the class.\\

    Hints:
    \begin{itemize}
    \item A decorator can be any callable, even a method
    \item You can override the \py{__call__} special method to make the object callable
    \end{itemize}

  \end{block}
\end{frame}


\end{document}
