\documentclass[14pt,compress,aspectratio=169]{beamer}


% Modified from: generic-ornate-15min-45min.de.tex
\mode<presentation>
{
  \usetheme{Madrid}
  \useoutertheme{infolines}
  \setbeamercovered{invisible}
}

\usepackage[english]{babel}
\usepackage[latin1]{inputenc}
%\usepackage{times}
\usepackage[T1]{fontenc}

% Taken from Fernando's slides.
\usepackage{ae,aecompl}
\usepackage{mathpazo,courier,euler}
\usepackage[scaled=.95]{helvet}

\definecolor{darkgreen}{rgb}{0,0.5,0}

\usepackage{listings}
\lstset{language=Python,
    basicstyle=\ttfamily\bfseries,
    commentstyle=\color{red}\itshape,
  stringstyle=\color{darkgreen},
  showstringspaces=false,
  keywordstyle=\color{blue}\bfseries}

%%%%%%%%%%%%%%%%%%%%%%%%%%%%%%%%%%%%%%%%%%%%%%%%%%%%%%%%%%%%%%%%%%%%%%
% Macros
\setbeamercolor{emphbar}{bg=blue!20, fg=black}
\newcommand{\emphbar}[1]
{\begin{beamercolorbox}[rounded=true]{emphbar}
      {#1}
 \end{beamercolorbox}
}
\newcounter{time}
\setcounter{time}{0}
\newcommand{\inctime}[1]{\addtocounter{time}{#1}{\tiny \thetime\ m}}

\newcommand{\typ}[1]{\textbf{\texttt{{#1}}}}


\newcommand{\kwrd}[1]{ \texttt{\textbf{\color{blue}{#1}}}  }

%%% This is from Fernando's setup.
% \usepackage{color}
% \definecolor{orange}{cmyk}{0,0.4,0.8,0.2}
% % Use and configure listings package for nicely formatted code
% \usepackage{listings}
% \lstset{
%    language=Python,
%    basicstyle=\small\ttfamily,
%    commentstyle=\ttfamily\color{blue},
%    stringstyle=\ttfamily\color{orange},
%    showstringspaces=false,
%    breaklines=true,
%    postbreak = \space\dots
% }

%\pgfdeclareimage[height=0.75cm]{iitmlogo}{iitmlogo}
%\logo{\pgfuseimage{iitmlogo}}


%% Delete this, if you do not want the table of contents to pop up at
%% the beginning of each subsection:
\AtBeginSubsection[]
{
  \begin{frame}<beamer>
    \frametitle{Outline}
    \tableofcontents[currentsection,currentsubsection]
  \end{frame}
}

\AtBeginSection[]
{
  \begin{frame}<beamer>
    \frametitle{Outline}
    \tableofcontents[currentsection,currentsubsection]
  \end{frame}
}

% If you wish to uncover everything in a step-wise fashion, uncomment
% the following command:
%\beamerdefaultoverlayspecification{<+->}

%\includeonlyframes{current,current1,current2,current3,current4,current5,current6}


\title[OOP: Miscellaneous]{Advanced Python}
\subtitle{Object Oriented Programming: miscellaneous}

\author[FOSSEE] {The FOSSEE Group}

\institute[IIT Bombay] {Department of Aerospace Engineering\\IIT Bombay}
\date[] {Mumbai, India}

\begin{document}

\begin{frame}
  \titlepage
\end{frame}

\begin{frame}
  \frametitle{Recap}
  \begin{itemize}
  \item Classes for encapsulation
  \item Inheritance
  \item \py{super,isinstance,issubclass}
  \item Containership
  \item Class attributes and methods
    \vspace*{0.3in}
  \item Look at some odds and ends
  \end{itemize}
\end{frame}

\begin{frame}[fragile]
  \frametitle{More on methods}
\begin{lstlisting}
class A:
    def f(self):
        print("A.f")

In []: a = A()
In []: a.f()
A.f
In []: A.f(a)

In []: A.f()
\end{lstlisting}
\end{frame}

\begin{frame}[fragile]
  \frametitle{More on methods}
\begin{lstlisting}
In []: type(a.f)
Out[]: method
In []: type(A.f)
Out[]: function
\end{lstlisting}
\end{frame}

\begin{frame}[fragile]
  \frametitle{Private and public attributes}
  \begin{itemize}
  \item No explicit support
  \item In practice, use \py{_attr} for private attributes/methods
  \end{itemize}
\begin{lstlisting}
class A:
    def __init__(self, x):
        self._x = x
    def _private_method(self):
        pass
\end{lstlisting}
\end{frame}

\begin{frame}[fragile]
  \frametitle{Private and public attributes}
  \begin{itemize}
  \item Access private methods/attributes inside class
  \item Avoid access from outside
  \end{itemize}

\begin{lstlisting}
In []: a = A()
In []: a._method() # Bad!
In []: a._x # Bad!
\end{lstlisting}
\end{frame}

\begin{frame}[fragile]
  \frametitle{Private and public attributes}
\begin{lstlisting}
class B:
    def _f(self):
        # OK!
        return self._x

    def func(self):
        # Also OK!
        return self._f()
\end{lstlisting}
\end{frame}

\begin{frame}
  \frametitle{Abstract classes}
  \begin{itemize}
  \item Used to specify an ``interface''
  \item Not meant to instantiate but subclass
  \item Useful to specify behavior
  \end{itemize}
\end{frame}

\begin{frame}[fragile]
  \frametitle{Abstract classes}
\begin{lstlisting}
class Abstract:
    def func(self):
        raise NotImplementedError('override')

class Derived(Abstract):
    def func(self):
        print("Hello world!")
\end{lstlisting}
  \begin{itemize}
  \item \py{Derived} is a concrete class
  \end{itemize}
\end{frame}

\begin{frame}[fragile]
  \frametitle{Modern abstract classes}
\begin{lstlisting}
from abc import ABC, abstractmethod

class Abstract(ABC):
    @abstractmethod
    def func(self):
        pass
\end{lstlisting}
  \begin{itemize}
  \item More elegant and simple
  \item Cannot instantiate \py{Abstract}
  \end{itemize}
\end{frame}


\begin{frame}
  \frametitle{Summary}
  \begin{itemize}
  \item Understanding methods a bit better
  \item Private/public methods/attributes
  \item Abstract and concrete classes
  \item Using the \py{abc} module
  \end{itemize}
\end{frame}


\end{document}
