\documentclass[14pt,compress,aspectratio=169]{beamer}


% Modified from: generic-ornate-15min-45min.de.tex
\mode<presentation>
{
  \usetheme{Warsaw}
  \useoutertheme{infolines}
  \setbeamercovered{invisible}
}

\usepackage[english]{babel}
\usepackage[latin1]{inputenc}
%\usepackage{times}
\usepackage[T1]{fontenc}

% Taken from Fernando's slides.
\usepackage{ae,aecompl}
\usepackage{mathpazo,courier,euler}
\usepackage[scaled=.95]{helvet}

\definecolor{darkgreen}{rgb}{0,0.5,0}

\usepackage{listings}
\lstset{language=Python,
    basicstyle=\ttfamily\bfseries,
    commentstyle=\color{red}\itshape,
  stringstyle=\color{darkgreen},
  showstringspaces=false,
  keywordstyle=\color{blue}\bfseries}

%%%%%%%%%%%%%%%%%%%%%%%%%%%%%%%%%%%%%%%%%%%%%%%%%%%%%%%%%%%%%%%%%%%%%%
% Macros
\setbeamercolor{emphbar}{bg=blue!20, fg=black}
\newcommand{\emphbar}[1]
{\begin{beamercolorbox}[rounded=true]{emphbar}
      {#1}
 \end{beamercolorbox}
}
\newcounter{time}
\setcounter{time}{0}
\newcommand{\inctime}[1]{\addtocounter{time}{#1}{\tiny \thetime\ m}}

\newcommand{\typ}[1]{\textbf{\texttt{{#1}}}}


\newcommand{\kwrd}[1]{ \texttt{\textbf{\color{blue}{#1}}}  }

%%% This is from Fernando's setup.
% \usepackage{color}
% \definecolor{orange}{cmyk}{0,0.4,0.8,0.2}
% % Use and configure listings package for nicely formatted code
% \usepackage{listings}
% \lstset{
%    language=Python,
%    basicstyle=\small\ttfamily,
%    commentstyle=\ttfamily\color{blue},
%    stringstyle=\ttfamily\color{orange},
%    showstringspaces=false,
%    breaklines=true,
%    postbreak = \space\dots
% }

%\pgfdeclareimage[height=0.75cm]{iitmlogo}{iitmlogo}
%\logo{\pgfuseimage{iitmlogo}}


%% Delete this, if you do not want the table of contents to pop up at
%% the beginning of each subsection:
\AtBeginSubsection[]
{
  \begin{frame}<beamer>
    \frametitle{Outline}
    \tableofcontents[currentsection,currentsubsection]
  \end{frame}
}

\AtBeginSection[]
{
  \begin{frame}<beamer>
    \frametitle{Outline}
    \tableofcontents[currentsection,currentsubsection]
  \end{frame}
}

% If you wish to uncover everything in a step-wise fashion, uncomment
% the following command:
%\beamerdefaultoverlayspecification{<+->}

%\includeonlyframes{current,current1,current2,current3,current4,current5,current6}


\title[Introduction to OOP]{Advanced Python}
\subtitle{Introduction to Object Oriented Programming }

\author[FOSSEE] {The FOSSEE Group}

\institute[IIT Bombay] {Department of Aerospace Engineering\\IIT Bombay}
\date[] {Mumbai, India}

\begin{document}

\begin{frame}
  \titlepage
\end{frame}

\begin{frame}
  \frametitle{Background and caveats}
  \begin{itemize}
  \item You don't need OOP for everything!
  \item Helps with larger programs
  \item Appreciate it, when you write more code
  \item Very useful to organize/modularize your code
  \end{itemize}
\end{frame}

\begin{frame}[fragile]
  \frametitle{What are the benefits?}
  \begin{itemize}
  \item Create new data types
  \item Create objects to simulate a real problem
  \item Makes problem solving more natural and elegant
  \item Can be easier to understand
  \item Allows for code-reuse
  \end{itemize}
\end{frame}


\begin{frame}[fragile]
  \frametitle{Example: Managing Talks}
  \begin{itemize}
  \item Want to manage list of talks at a conference
  \end{itemize}
  \begin{lstlisting}
talk = {'Speaker': 'Guido van Rossum',
        'Title': 'The History of Python'
        'Tags': 'python,history,C,advanced'}

def get_first_name(talk):
    return talk['Speaker'].split()[0]

def get_tags(talk):
    return talk['Tags'].split(',')
  \end{lstlisting}
\end{frame}

\begin{frame}
  \frametitle{Issues}
  \begin{itemize}
  \item Not convenient to handle large number of talks
  \item Data separate from functions to manipulate talk
  \item Must pass \lstinline{talk} to each function
  \item No clear organization
  \item What if we had different kinds of talks?
  \end{itemize}
\end{frame}


\begin{frame}[fragile]
  \frametitle{Creating a \typ{class}}
  \begin{lstlisting}
class Talk:
    """A class for the Talks."""
    def __init__(self, speaker, title, tags):
        self.speaker = speaker
        self.title = title
        self.tags = tags
  \end{lstlisting}
  \begin{itemize}
  \item \lstinline{class} is a keyword
  \item \lstinline{Talk} is the name of the class
  \item Notice the class doc-string
  \item Notice the indentation
  \end{itemize}
\end{frame}

\begin{frame}[fragile]
  \frametitle{methods and \lstinline{self}}
  \begin{lstlisting}
class Talk:
    # ...
    def get_speaker_firstname(self):
        return self.speaker.split()[0]

    def get_tags(self):
        return self.tags.split(',')
  \end{lstlisting}
  \begin{itemize}
  \item \lstinline{self} is a reference to the object itself
  \item The name \lstinline{self} is a convention
  \item Instance variables: \lstinline{self.name}
  \end{itemize}
\end{frame}

\begin{frame}[fragile, plain]
  \frametitle{Exercise: type this out}
  \vspace*{-0.1in}
  \begin{lstlisting}
class Talk:
    """A class for the Talks."""
    def __init__(self, speaker, title, tags):
        self.speaker = speaker
        self.title = title
        self.tags = tags

    def get_speaker_firstname(self):
        return self.speaker.split()[0]

    def get_tags(self):
        return self.tags.split(',')
  \end{lstlisting}
\end{frame}


\begin{frame}[fragile]
  \frametitle{Instantiating a class to create objects}
  \begin{itemize}
  \item Creating objects or instances of a class is simple
  \item Call class, with arguments required by \lstinline{__init__}
  \end{itemize}
  \begin{lstlisting}
In []: bdfl = Talk('Guido van Rossum',
  ...:             'The History of Python',
  ...:             'python,history,C,advanced')
  \end{lstlisting}
  \begin{itemize}
  \item We can now call the methods of the Class
  \end{itemize}
  \begin{lstlisting}
In []: bdfl.get_tags()
In []: bdfl.get_speaker_firstname()
  \end{lstlisting}
\end{frame}

\begin{frame}[fragile]
  \frametitle{\lstinline{__init__} method}
  \begin{itemize}
  \item A special method
  \item Called every time an instance of the class is created
  \end{itemize}
  \begin{lstlisting}
In []: print(bdfl.speaker)
In []: print(bdfl.tags)
In []: print(bdfl.title)
  \end{lstlisting}
\end{frame}

\begin{frame}[fragile]
  \frametitle{Creating more instances}
  \begin{lstlisting}
In []: talk = Talk('Arun K',
  ...:      'Python in biosciences', 'python,bio')
In []: talk.get_tags()
In []: talk.tags

In []: type(talk)
Out[]: __main__.Talk
In []: type(bdfl)
\end{lstlisting}
\begin{itemize}
\item Note that \lstinline{talk} and \lstinline{bdfl} are different objects
\item Different values but same type
\item \lstinline{Talk} \alert{encapsulates} the data and behavior of a talk
\end{itemize}
\end{frame}

\begin{frame}[fragile]
  \frametitle{Summary}
  \begin{itemize}
  \item Introduction to Object Oriented Programming
  \item A simple example
  \item Defining a \lstinline{class}
  \item Methods and attributes: encapsulation
  \end{itemize}
\end{frame}



\end{document}
