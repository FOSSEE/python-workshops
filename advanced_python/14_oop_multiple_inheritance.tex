\documentclass[14pt,compress,aspectratio=169]{beamer}

\usepackage{hyperref}

% Modified from: generic-ornate-15min-45min.de.tex
\mode<presentation>
{
  \usetheme{Warsaw}
  \useoutertheme{infolines}
  \setbeamercovered{invisible}
}

\usepackage[english]{babel}
\usepackage[latin1]{inputenc}
%\usepackage{times}
\usepackage[T1]{fontenc}

% Taken from Fernando's slides.
\usepackage{ae,aecompl}
\usepackage{mathpazo,courier,euler}
\usepackage[scaled=.95]{helvet}

\definecolor{darkgreen}{rgb}{0,0.5,0}

\usepackage{listings}
\lstset{language=Python,
    basicstyle=\ttfamily\bfseries,
    commentstyle=\color{red}\itshape,
  stringstyle=\color{darkgreen},
  showstringspaces=false,
  keywordstyle=\color{blue}\bfseries}

%%%%%%%%%%%%%%%%%%%%%%%%%%%%%%%%%%%%%%%%%%%%%%%%%%%%%%%%%%%%%%%%%%%%%%
% Macros
\setbeamercolor{emphbar}{bg=blue!20, fg=black}
\newcommand{\emphbar}[1]
{\begin{beamercolorbox}[rounded=true]{emphbar}
      {#1}
 \end{beamercolorbox}
}
\newcounter{time}
\setcounter{time}{0}
\newcommand{\inctime}[1]{\addtocounter{time}{#1}{\tiny \thetime\ m}}

\newcommand{\typ}[1]{\textbf{\texttt{{#1}}}}


\newcommand{\kwrd}[1]{ \texttt{\textbf{\color{blue}{#1}}}  }

%%% This is from Fernando's setup.
% \usepackage{color}
% \definecolor{orange}{cmyk}{0,0.4,0.8,0.2}
% % Use and configure listings package for nicely formatted code
% \usepackage{listings}
% \lstset{
%    language=Python,
%    basicstyle=\small\ttfamily,
%    commentstyle=\ttfamily\color{blue},
%    stringstyle=\ttfamily\color{orange},
%    showstringspaces=false,
%    breaklines=true,
%    postbreak = \space\dots
% }

%\pgfdeclareimage[height=0.75cm]{iitmlogo}{iitmlogo}
%\logo{\pgfuseimage{iitmlogo}}


%% Delete this, if you do not want the table of contents to pop up at
%% the beginning of each subsection:
\AtBeginSubsection[]
{
  \begin{frame}<beamer>
    \frametitle{Outline}
    \tableofcontents[currentsection,currentsubsection]
  \end{frame}
}

\AtBeginSection[]
{
  \begin{frame}<beamer>
    \frametitle{Outline}
    \tableofcontents[currentsection,currentsubsection]
  \end{frame}
}

% If you wish to uncover everything in a step-wise fashion, uncomment
% the following command:
%\beamerdefaultoverlayspecification{<+->}

%\includeonlyframes{current,current1,current2,current3,current4,current5,current6}



\title[OOP: multiple inheritance]{Advanced Python}
\subtitle{Object Oriented Programming: Multiple Inheritance}

\author[FOSSEE] {The FOSSEE Group}

\institute[IIT Bombay] {Department of Aerospace Engineering\\IIT Bombay}
\date[] {Mumbai, India}

\begin{document}

\begin{frame}
  \titlepage
\end{frame}

\begin{frame}[fragile]
  \frametitle{Overview}
  \begin{itemize}
  \item Thus far, single inheritance
  \end{itemize}
  \begin{lstlisting}
    class Animal:
        def greet(self): ...

    class Cat(Animal):
        def greet(self): ...

    class Dog(Animal):
        def greet(self): ...
\end{lstlisting}
\end{frame}

\begin{frame}[fragile]
  \frametitle{Motivation}
  \begin{itemize}
  \item What if you wanted two parents?
  \end{itemize}

\begin{lstlisting}
class Animal:
    def eat(self): pass
  \end{lstlisting}
  \pause
  \begin{lstlisting}
class Mammal(Animal):
    def hair_color(self): pass
  \end{lstlisting}
  \pause
\begin{lstlisting}
class FlyingAnimal(Animal):
    def fly(self): pass

\end{lstlisting}
\end{frame}

\begin{frame}[fragile]
  \frametitle{Motivation \ldots}
\begin{lstlisting}
class Bat(Mammal, FlyingAnimal):
    pass

In []: b = Bat()

In []: b.eat(), b.hair_color(), b.fly()
\end{lstlisting}
\vspace*{0.25in}
\footnotesize{Example from \href{https://en.wikipedia.org/wiki/Virtual_inheritance}{Wikipedia}}
\end{frame}


\begin{frame}
  \frametitle{Observations}
  \begin{itemize}
  \item \py{Mammal} is an \py{Animal}
  \item \py{FlyingAnimal} is an \py{Animal}
  \item \py{FlyingAnimal} is not a \py{Mammal}
  \item \py{Bat} is a \py{Mammal} and a \py{FlyingAnimal}
  \end{itemize}
\end{frame}

\begin{frame}[fragile, plain]
  \frametitle{Multiple inheritance and super}
  \small
  \vspace*{-0.1in}
  \begin{lstlisting}
class Base:
    def __init__(self):
        print('Base')

class A(Base):
    def __init__(self):
        print('A')
        super().__init__()

class B(Base):
    def __init__(self):
        print('B')
        super().__init__()

  \end{lstlisting}

\end{frame}

\begin{frame}[fragile]
  \frametitle{Multiple inheritance and super}
  \small
  \begin{lstlisting}
class C(A, B):
    def __init__(self):
        print('C')
        super().__init__()

In []: c = C()
C
A
B
Base
\end{lstlisting}

\end{frame}


\begin{frame}[fragile]
  \frametitle{Changing the order of parents}
  \small
  \begin{lstlisting}
class C1(B, A):
    def __init__(self):
        print('C')
        super().__init__()

In []: c1 = C1()
C1
B
A
Base
\end{lstlisting}
\end{frame}

\begin{frame}[fragile]
  \frametitle{\py{mro}:  method resolution order}
  \begin{itemize}
  \item \py{mro} method shows order of calls
  \item A method on the class object, not the instance
  \end{itemize}
  \begin{lstlisting}
In []: C.mro()
Out[]: [__main__.C, __main__.A, __main__.B,
        __main__.Base, object]

In []: C1.mro()
Out[]: [__main__.C1, __main__.B, __main__.A,
        __main__.Base, object]
  \end{lstlisting}
\end{frame}

\begin{frame}[fragile, plain]
  \frametitle{Another example}
  \small
  \vspace*{-0.1in}
  \begin{lstlisting}
class A:
    def __init__(self):
        print('A')
class B:
    def __init__(self):
        print('B')

class C(A, B):
    def __init__(self):
        print('C')
        super().__init__()

In []: c = C()
C
A
\end{lstlisting}

\end{frame}

\begin{frame}
  \frametitle{Observations}
  \begin{itemize}
  \item \py{A} and \py{B} did not call \py{super}
  \item \py{super} is what calls the parent \py{__init__}
  \item The order of calls is determined by \py{mro}
  \item This works for any method, not just \py{__init__}
  \end{itemize}
\end{frame}

\begin{frame}[fragile]
  \frametitle{Question}
  \begin{itemize}
  \item What if \py{C} did not call \py{super}?
  \end{itemize}
  \begin{lstlisting}
class C(A, B):
    def __init__(self):
        print('C')

In []: c = C()
  \end{lstlisting}
\end{frame}

\begin{frame}
  \frametitle{Learning more}
  \begin{itemize}
  \item See: \url{docs.python.org/3.6/tutorial/classes.html}
  \item See: \url{www.python.org/download/releases/2.3/mro/}
  \end{itemize}
\end{frame}

\begin{frame}
  \frametitle{Summary}
  \begin{itemize}
  \item Multiple base classes
  \item Important of \py{super}
  \item \py{mro} determines order of calls
  \end{itemize}
\end{frame}


\end{document}
