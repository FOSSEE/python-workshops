\documentclass[14pt,compress,aspectratio=169]{beamer}


% Modified from: generic-ornate-15min-45min.de.tex
\mode<presentation>
{
  \usetheme{Madrid}
  \useoutertheme{infolines}
  \setbeamercovered{invisible}
}

\usepackage[english]{babel}
\usepackage[latin1]{inputenc}
%\usepackage{times}
\usepackage[T1]{fontenc}

% Taken from Fernando's slides.
\usepackage{ae,aecompl}
\usepackage{mathpazo,courier,euler}
\usepackage[scaled=.95]{helvet}

\definecolor{darkgreen}{rgb}{0,0.5,0}

\usepackage{listings}
\lstset{language=Python,
    basicstyle=\ttfamily\bfseries,
    commentstyle=\color{red}\itshape,
  stringstyle=\color{darkgreen},
  showstringspaces=false,
  keywordstyle=\color{blue}\bfseries}

%%%%%%%%%%%%%%%%%%%%%%%%%%%%%%%%%%%%%%%%%%%%%%%%%%%%%%%%%%%%%%%%%%%%%%
% Macros
\setbeamercolor{emphbar}{bg=blue!20, fg=black}
\newcommand{\emphbar}[1]
{\begin{beamercolorbox}[rounded=true]{emphbar}
      {#1}
 \end{beamercolorbox}
}
\newcounter{time}
\setcounter{time}{0}
\newcommand{\inctime}[1]{\addtocounter{time}{#1}{\tiny \thetime\ m}}

\newcommand{\typ}[1]{\textbf{\texttt{{#1}}}}


\newcommand{\kwrd}[1]{ \texttt{\textbf{\color{blue}{#1}}}  }

%%% This is from Fernando's setup.
% \usepackage{color}
% \definecolor{orange}{cmyk}{0,0.4,0.8,0.2}
% % Use and configure listings package for nicely formatted code
% \usepackage{listings}
% \lstset{
%    language=Python,
%    basicstyle=\small\ttfamily,
%    commentstyle=\ttfamily\color{blue},
%    stringstyle=\ttfamily\color{orange},
%    showstringspaces=false,
%    breaklines=true,
%    postbreak = \space\dots
% }

%\pgfdeclareimage[height=0.75cm]{iitmlogo}{iitmlogo}
%\logo{\pgfuseimage{iitmlogo}}


%% Delete this, if you do not want the table of contents to pop up at
%% the beginning of each subsection:
\AtBeginSubsection[]
{
  \begin{frame}<beamer>
    \frametitle{Outline}
    \tableofcontents[currentsection,currentsubsection]
  \end{frame}
}

\AtBeginSection[]
{
  \begin{frame}<beamer>
    \frametitle{Outline}
    \tableofcontents[currentsection,currentsubsection]
  \end{frame}
}

% If you wish to uncover everything in a step-wise fashion, uncomment
% the following command:
%\beamerdefaultoverlayspecification{<+->}

%\includeonlyframes{current,current1,current2,current3,current4,current5,current6}


\title[OOP: Class Attributes]{Advanced Python}
\subtitle{Object Oriented Programming: class attributes}

\author[FOSSEE] {The FOSSEE Group}

\institute[IIT Bombay] {Department of Aerospace Engineering\\IIT Bombay}
\date[] {Mumbai, India}

\begin{document}

\begin{frame}
  \titlepage
\end{frame}

\begin{frame}
  \frametitle{Recap}
  \begin{itemize}
  \item Classes for encapsulation
  \item Inheritance
  \item \py{super,isinstance,issubclass}
  \item Containership
    \vspace*{0.3in}
  \item Look at class attributes/methods
  \end{itemize}
\end{frame}


\begin{frame}[fragile]
  \frametitle{Class/instance attributes}
  \begin{itemize}
  \item So far: instance attributes
  \end{itemize}
\begin{lstlisting}
class A:
    x = 1

In []: a = A()
In []: a.x
Out[]: 1

In []: b = A()
In []: assert b.x == a.x
\end{lstlisting}
\end{frame}

\begin{frame}[fragile]
  \frametitle{Class/instance attributes}
\begin{lstlisting}
class A:
    x = 1

In []: a, b = A(), A()

In []: A.x = 2
In []: b.x
Out[]: 2
\end{lstlisting}
\end{frame}

\begin{frame}[fragile]
  \frametitle{Class/instance attributes: beware}
\begin{lstlisting}
class A:
    x = 1

In []: a = A()
In []: a.x = 1
In []: A.x = 2
In []: a.x
Out[]: ??
\end{lstlisting}
\end{frame}

\begin{frame}
  \frametitle{Understanding the scope}
  \begin{itemize}
  \item If \py{a.x} is not an instance attribute
  \item Python looks in the class for the attribute!
    \pause
    \vspace*{0.25in}
  \item So \py{a.x = 1} sets an instance attribute!
  \end{itemize}
\end{frame}

\begin{frame}[fragile,fragile]
  \frametitle{Class attribute scope}
\begin{lstlisting}
class A:
    x = 1
    def f(self):
        return self.x + 1

In []: a = A()
In []: a.f()
\end{lstlisting}
  \pause
\begin{lstlisting}
Out[]: 2
\end{lstlisting}
\end{frame}


\begin{frame}[fragile]
  \frametitle{Class methods}
  \begin{itemize}
  \item Can we have methods on the class?
  \end{itemize}
\begin{lstlisting}
class A:
    @classmethod
    def method(cls, x):
        print(cls, x)

In []: A.method(1)
<class '__main__.A'> 1
\end{lstlisting}
\end{frame}

\begin{frame}
  \frametitle{Class methods}
  \begin{itemize}
  \item \py{@classmethod} is a special decorator
  \item Makes the method a class method
  \item Implicit argument is the class (and not the instance)
  \item Useful when you don't want to construct the object
  \end{itemize}
\end{frame}

\begin{frame}
  \frametitle{Summary}
  \begin{itemize}
  \item Class attributes
  \item Class methods and the \py{@classmethod} decorator
  \end{itemize}
\end{frame}

%%%%%%%%%%%%%%%%%%%%%%%%%%%%%%%%%

\begin{frame}[plain, fragile]
  \frametitle{Exercise: Point class attribute}
  \begin{block}{}
    Create a simple \py{Point} class with class attributes \py{x, y} that
    default to 0.0.
  \end{block}

\begin{lstlisting}
In []: p = Point()
In []: p.x, p.y
Out[]: (0.0, 0.0)
\end{lstlisting}
\end{frame}


\begin{frame}[plain, fragile]
  \frametitle{Solution}
\begin{lstlisting}
class Point:
    x = 0.0
    y = 0.0
\end{lstlisting}
\end{frame}

\begin{frame}[plain, fragile]
  \frametitle{Exercise: classmethod}
  \begin{block}{}
    Create a simple \py{Point} class with instance attributes \py{x, y} but
    add a class method \py{polar(r, theta)} that takes coordinates in $(r,
    \theta)$ form but returns a suitable point. Remember that, $x=r
    \cos(\theta), y=r\sin(\theta)$. For example:
  \end{block}

\begin{lstlisting}
In []: p = Point.polar(1.0, 0.0)
In []: p.x, p.y
Out[]: (1.0, 0.0)
\end{lstlisting}

\end{frame}


\begin{frame}[plain, fragile]
  \frametitle{Solution}
\begin{lstlisting}
from math import sin, cos
class Point:
    def __init__(self, x=0.0, y=0.0):
        self.x = x
        self.y = y
    @classmethod
    def polar(cls, r, theta):
        x = r*cos(theta)
        y = r*sin(theta)
        return cls(x, y)

\end{lstlisting}
\end{frame}



\end{document}
