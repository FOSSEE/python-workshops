\documentclass{article}
\usepackage{graphicx}
\usepackage[landscape]{geometry}
\usepackage[pdftex]{color}
\usepackage{url}
\usepackage{multicol}
\usepackage{amsmath}
\usepackage{amsfonts}
\newcommand{\ex}{\color{blue}}
\pagestyle{empty}
\advance\topmargin-.9in
\advance\textheight2in
\advance\textwidth3.0in
\advance\oddsidemargin-1.45in
\advance\evensidemargin-1.45in
\parindent0pt
\parskip2pt

\newcommand{\hr}{\centerline{\rule{3.5in}{1pt}}}
\newcommand{\skipin}{\hspace*{12pt}}
\newcommand{\typ}[1]{\lstinline{#1}}

\usepackage{color}
\definecolor{darkgreen}{rgb}{0,0.5,0}

\usepackage{listings}
\lstset{language=Python,
    basicstyle=\ttfamily\bfseries,
    commentstyle=\color{red}\itshape,
  stringstyle=\color{darkgreen},
  showstringspaces=false,
  keywordstyle=\color{blue}\bfseries}

\begin{document}
\begin{multicols*}{3}
\begin{center}
\textbf{Scientific Computing with Python}\\
\textbf{Quick Reference}\\
\textbf{SciPy India 2016}\\
\end{center}
\vspace{-2ex}

%*********************************************
 \hr\textbf{Starting up}

To start \lstinline|ipython| with \lstinline|pylab|:\\
\lstinline|    $ ipython --pylab| %$

To exit: \lstinline|^D| (Ctrl-d)

To break from loops: \lstinline|^C| (Ctrl-c)

%*********************************************
\hr\textbf{Simple plotting}

Creating a linear array:\\
{\ex \lstinline|    x = linspace(0, 2*pi, 50)|}

Plotting two variables:\\
{\ex \lstinline|    plot(x, sin(x))|}

Plotting two lists of equal length x, y:\\
{\ex \lstinline|    plot(x, y)|}

Plots with colors:\\
{\ex \lstinline|    plot(x, sin(x), 'b')|} gives a blue line

Line style and markers:\\
{\ex \lstinline|    plot(x, sin(x), '--')|} gives a dashed line\\
{\ex \lstinline|'.'|} --  a point marker, {\ex \lstinline|'o'|} -- a circle marker

Labels:\\
{\ex \lstinline|    xlabel('x')|} and {\ex \lstinline|ylabel('sin(x)')|}

Title (\lstinline|pylab| accepts \TeX~in any text expression):\\
{\ex \lstinline|    title(r'$\sigma$ vs. $\sin(\sigma)$')|}

Legend:\\
{\ex \lstinline|    legend(['sin(x)'], loc=center)|}\\
{\ex \lstinline|    legend(['sin(x)', 'cos(x)'])|}\\
If \lstinline|loc| is not specified, \lstinline|best| is used

Annotate:\\
{\ex \lstinline|    annotate('annotation string', xy=(1.5, 1))|}

Saving figures:\\
{\ex \lstinline|    savefig('sinusoids.png')|}

Axes lengths:\\
Get the axes lengths:\\
{\ex \lstinline|    xmin, xmax = xlim()|}\\
{\ex \lstinline|    ymin, ymax = ylim()|}\\
Set the axes lengths:\\
{\ex \lstinline|    xlim(-2*pi, 2*pi)|}\\
{\ex \lstinline|    ylim(-1.2, 1.2)|}

Miscellaneous:\\
{\ex \lstinline|    clf()|} to clear the plot area\\
{\ex \lstinline|    close()|} to close the figure

%*********************************************
\hr\textbf{IPython tips}

\typ{TAB} completes commands\\
History: Up, down arrows or (\typ{Ctrl-p}/\typ{Ctrl-n})\\
Search: \typ{Ctrl-r} and start typing\\
\typ{Ctrl-a}: go to start of line\\
\typ{Ctrl-e}: go to end of line\\
\typ{Ctrl-k}: kill to end of line\\
Help: \typ{object?}\\
Source Code: \typ{object??}\\
Time execution of expression/statement: \typ{\%timeit}

%*********************************************
\hr\textbf{Saving and Running scripts}

{\ex \lstinline|    %hist|} returns history of commands used. \\
To save a set of lines, say 14-18, 20, 22, to \lstinline|sample.py| \\
{\ex \lstinline|    %save sample.py 14-18 20 22|}\\
To run \lstinline|sample.py| \\
{\ex \lstinline|    %run -i sample.py|}

%*********************************************
 \hr\textbf{Reading from files}

\lstinline|filename.txt| is a file with float data.
Using loadtxt: \\
{\ex \lstinline|X = loadtxt('filename.txt')|} \\
X is an array with all the data from \lstinline|filename.txt|\\
{\ex \lstinline|X,Y = loadtxt('filename.txt', unpack=True)|} \\
X,Y contain each column of the data\\
{\ex \lstinline|X = loadtxt('filename.txt', delimiter=';')|} \\
when ';' delmits the columns of data

%*********************************************
\hr\textbf{Statistical operations}

{\ex \lstinline|mean|, \lstinline|median|, \lstinline|std|}

%*********************************************
\hr\textbf{NumPy Arrays}

Fixed size: \typ{arr.size}\\
Homogeneous: \typ{arr.dtype}\\
Extent along each dimension: \typ{arr.shape}\\
Bytes per element: \typ{arr.itemsize}

%*********************************************
\hr\textbf{Array Creation}

{\ex \lstinline|C = array([[11,12,13], [21,22,23], [31,32,33]])|}\\
{\ex \lstinline|C.shape|} shape--- rows \& cols\\
{\ex \lstinline|C.dtype|} data type\\
{\ex \lstinline|B = ones_like(C)|} array of ones; same shape, dtype as C\\
\skipin similarly \lstinline|zeros_like, empty_like| \\
{\ex \lstinline|A = ones((3,2))|} array of ones of shape (3,2)\\
\skipin similarly \lstinline|zeros, empty|\\
{\ex \lstinline|I = identity(3)|} identity matrix of size 3x3\\
{\ex \lstinline|x, y = mgrid[0:3, 0:5]|} mesh-grid of size 3x5\\
{\ex \lstinline|x, y = ogrid[0:3, 0:5]|} open mesh-grid of size 3x5

%*********************************************
\hr\textbf{Accessing \& Changing elements}

{\ex \lstinline|C[1, 2]|} gets third element of second row\\
\skipin \textbf{Note:} Indexing starts from 0. \\
{\ex \lstinline|C[1]|} gets the second row \\
{\ex \lstinline|C[1,:]|} same as above (`:' implies all columns)\\
{\ex \lstinline|C[:,1]|} gets the second column (`:' implies all rows)\\
{\ex \lstinline|C[0:2,:]|} or {\ex \lstinline|C[:2,:]|} gets $1^{st}, 2^{nd}$ rows; all cols\\
{\ex \lstinline|C[1:3,:]|} or {\ex \lstinline|C[1:,:]|} gets $2^{nd}, 3^{rd}$
rows; all cols\\
{\ex \lstinline|C[0:3:2,:]|} or {\ex \lstinline|C[::2,:]|} gets $1^{st}, 3^{rd}$
rows; all cols

%*********************************************
\hr\textbf{Matrix Operations}

For matrices A and B of equal shapes:\\
{\ex \lstinline|A.T|} transpose\\
{\ex \lstinline|sum(A)|} sum of all elements\\
{\ex \lstinline|A+B|} addition\\
{\ex \lstinline|A*B|} element wise product\\
{\ex \lstinline|dot(A, B)|} Matrix multiplication\\
{\ex \lstinline|inv(A)|} inverse, {\ex \lstinline|det(A)|} determinant\\
{\ex \lstinline|eig(A)|} eigen values and vectors\\
{\ex \lstinline|norm(A)|} norm\\
{\ex \lstinline|svd(A)|} singular value decomposition

%*********************************************
\hr\textbf{Least Square Fit}

To get the least square fit of $L$ vs. $tsq$:\\
{\ex \lstinline|A = array([L, ones_like(L)])|}\\
{\ex \lstinline|A = A.T|} vandermonde matrix\\
{\ex \lstinline|result = lstsq(A, tsq)|}\\
{\ex \lstinline|coef = result[0]|} coefficients\\
{\ex \lstinline|Tline = coef[0]*L + coef[1]|}

%*********************************************
\hr\textbf{Solving Linear Equations}

{\ex \lstinline|A = array([[3,2,-1], [2,-2,4], [-1, 0.5, -1]])|}\\
\skipin coefficient array\\
{\ex \lstinline|b = array([1, -2, 0])|} constant array\\
{\ex \lstinline|x = solve(A, b)|} the required solution

Checking the solution:\\
{\ex \lstinline|Ax = dot(A,x)|} matrix multiplication of A and x\\
{\ex \lstinline|allclose(Ax, b)|} check the closeness of Ax, b

%*********************************************
\hr\textbf{Roots of Polynomials}

{\ex \lstinline|coeffs = [1, 6, 13]|} coefficients in descending order\\
{\ex \lstinline|roots(coeffs)|} returns complex roots of the polynomial
\hr
\pagebreak

%*********************************************
\hr\textbf{Roots of non-linear equations}

{\ex \lstinline|from scipy.optimize import fsolve|}\\
\skipin \lstinline|fsolve| is not in \lstinline|pylab|\\
\skipin we import from \lstinline|scipy.optimize| \\
We wish to find the roots of $f(x)=sin(x)+cos(x)^2$
\vspace{-8pt}
\begin{lstlisting}
def f(x):
    return sin(x)+cos(x)**2
\end{lstlisting}
\vspace{-8pt}
{\ex \lstinline|fsolve(f, 0)|} \\
\skipin arguments are function name and initial estimate

%*********************************************
\hr\textbf{ODE}

To solve the ODE below:\\
$\frac{dy}{dt} = ky(L-y)$, L = 25000, k = 0.00003, y(0) = 250\\
\vspace{-8pt}
\begin{lstlisting}
def f(y, t):
    k, L = 0.00003, 25000
    return k*y*(L-y)
\end{lstlisting}
\vspace{-8pt}
{\ex \lstinline|t = linspace(0, 12, 60)|} time over which to solve ODE\\
{\ex \lstinline|y0 = 250|} initial conditions\\
{\ex \lstinline|from scipy.integrate import odeint|}\\
{\ex \lstinline|y = odeint(f, y0, t)|}

\hr\textbf{FFT}

{\ex \lstinline|t = linspace(0, 2*pi, 500)|}\\
{\ex \lstinline|y = sin(4*pi*t)|} a sinusoidal signal\\
{\ex \lstinline|f = fft(y)|}\\
{\ex \lstinline|freq = fftfreq(500, t[1] - t[0])|}\\
{\ex \lstinline|plot(freq[:250], abs(f)[:250])|}

\hr\textbf{Random Numbers}

{\ex \typ{from numpy import random}}\\
{\ex \typ{random.random}}: Uniform deviates in $[0, 1)$\\
{\ex \typ{random.normal}}: Random samples -- Gaussian dist.\\
{\ex \typ{random.normal}}: Random samples -- Gaussian dist.\\
{\ex \typ{x = random.random(size=1000)}}\\
{\ex \typ{x,y = random.normal(size=(2,1000))}}

\hr\textbf{Record Arrays}

{\ex \typ{typ = [('id', int), ('x', float)]}}\\
{\ex \typ{rec = numpy.zeros(10, dtype=typ)}}\\
{\ex \typ{rec['id'] = range(10)}}\\
{\ex \typ{rec['x'] = random.random(size=10)}}\\

{\ex \typ{data = csv2rec('data.csv')}} from csv to record array

\hr\textbf{Basic image processing}

{\ex \typ{from pylab import imread, imshow, colorbar}}\\
{\ex \typ{a = imread('lena.png')}}: a is a NumPy array\\
{\ex \typ{imshow(a)}}\\
{\ex \typ{colorbar()}}\\

\hr\textbf{3D plotting with Mayavi's \typ{mlab}}

{\ex \typ{from mayavi import mlab}}  Ready to Go!\\
{\ex \typ{mlab.test_<TAB>}}  Test Functions\\
{\ex \typ{mlab.points3d(x, y, z)}}  Plot points in 3D\\
{\ex \typ{mlab.contour3d(x*x*0.5 + y*y + z*z*2)}}  3D contours\\
{\ex \typ{mlab.gcf}} get current figure\\
{\ex \typ{mlab.savefig}} save current figure\\
{\ex \typ{mlab.figure}} switch figure or new figure\\
{\ex \typ{mlab.axes}} create axes\\
{\ex \typ{mlab.outline}} create outline\\
{\ex \typ{mlab.title}} add title\\
{\ex \typ{mlab.xlabel, ylabel, zlabel}} labels\\
{\ex \typ{mlab.colorbar}} Add colorbar\\
{\ex \typ{mlab.scalarbar}} Add colorbar for scalar color mapping\\
{\ex \typ{mlab.vectorbar}} Add colorbar for vector color mapping\\
{\ex \typ{mlab.show}} show figure (standalone scripts)

\end{multicols*}

\end{document}
