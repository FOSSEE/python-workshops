
\documentclass[14pt,compress]{beamer}



% Modified from: generic-ornate-15min-45min.de.tex
\mode<presentation>
{
  \usetheme{Warsaw}
  \useoutertheme{infolines}
  \setbeamercovered{invisible}
}

\usepackage[english]{babel}
\usepackage[latin1]{inputenc}
%\usepackage{times}
\usepackage[T1]{fontenc}

% Taken from Fernando's slides.
\usepackage{ae,aecompl}
\usepackage{mathpazo,courier,euler}
\usepackage[scaled=.95]{helvet}

\definecolor{darkgreen}{rgb}{0,0.5,0}

\usepackage{listings}
\lstset{language=Python,
    basicstyle=\ttfamily\bfseries,
    commentstyle=\color{red}\itshape,
  stringstyle=\color{darkgreen},
  showstringspaces=false,
  keywordstyle=\color{blue}\bfseries}

%%%%%%%%%%%%%%%%%%%%%%%%%%%%%%%%%%%%%%%%%%%%%%%%%%%%%%%%%%%%%%%%%%%%%%
% Macros
\setbeamercolor{emphbar}{bg=blue!20, fg=black}
\newcommand{\emphbar}[1]
{\begin{beamercolorbox}[rounded=true]{emphbar}
      {#1}
 \end{beamercolorbox}
}
\newcounter{time}
\setcounter{time}{0}
\newcommand{\inctime}[1]{\addtocounter{time}{#1}{\tiny \thetime\ m}}

\newcommand{\typ}[1]{\textbf{\texttt{{#1}}}}


\newcommand{\kwrd}[1]{ \texttt{\textbf{\color{blue}{#1}}}  }

%%% This is from Fernando's setup.
% \usepackage{color}
% \definecolor{orange}{cmyk}{0,0.4,0.8,0.2}
% % Use and configure listings package for nicely formatted code
% \usepackage{listings}
% \lstset{
%    language=Python,
%    basicstyle=\small\ttfamily,
%    commentstyle=\ttfamily\color{blue},
%    stringstyle=\ttfamily\color{orange},
%    showstringspaces=false,
%    breaklines=true,
%    postbreak = \space\dots
% }

%\pgfdeclareimage[height=0.75cm]{iitmlogo}{iitmlogo}
%\logo{\pgfuseimage{iitmlogo}}


%% Delete this, if you do not want the table of contents to pop up at
%% the beginning of each subsection:
\AtBeginSubsection[]
{
  \begin{frame}<beamer>
    \frametitle{Outline}
    \tableofcontents[currentsection,currentsubsection]
  \end{frame}
}

\AtBeginSection[]
{
  \begin{frame}<beamer>
    \frametitle{Outline}
    \tableofcontents[currentsection,currentsubsection]
  \end{frame}
}

% If you wish to uncover everything in a step-wise fashion, uncomment
% the following command:
%\beamerdefaultoverlayspecification{<+->}

%\includeonlyframes{current,current1,current2,current3,current4,current5,current6}



%%%%%%%%%%%%%%%%%%%%%%%%%%%%%%%%%%%%%%%%%%%%%%%%%%%%%%%%%%%%%%%%%%%%%%
% Title page
\title[Testing]{Unit Testing in Python}

\author[FOSSEE] {Prabhu Ramachandran}

\institute[FOSSEE -- IITB] {Department of Aerospace Engineering\\IIT Bombay}
\date[] {Mumbai, India}

%%%%%%%%%%%%%%%%%%%%%%%%%%%%%%%%%%%%%%%%%%%%%%%%%%%%%%%%%%%%%%%%%%%%%%
% DOCUMENT STARTS
\begin{document}

\begin{frame}
  \titlepage
\end{frame}

\begin{frame}
  \frametitle{Outline}
  \tableofcontents
  % You might wish to add the option [pausesections]
\end{frame}

\section{Introduction}

\begin{frame}
  \frametitle{Objectives}
  At the end of this session, you will be able to:
  \begin{itemize}
  \item Write your code using the TDD paradigm
  \item Use the \typ{pytest} module to test your code
  \item Run your tests on \url{travis-ci.org}
  \item Look at test coverage of your code
  \item Look at automatic coverage with \url{codecov.io}
  \end{itemize}
\end{frame}

\begin{frame}
  \frametitle{Why?  What?}
  \begin{itemize}
  \item Do you think we make bridges without testing them?
    \pause
    \vspace*{1in}
  \item Why should software be any different?
  \end{itemize}
\end{frame}

\begin{frame}
  \frametitle{So what?}
  \begin{itemize}
  \item Easy to develop code
  \item Easy to refactor/improve code
  \item Confidence in code base
  \item Make you feel warm all over!
  \end{itemize}
\end{frame}


\begin{frame}[plain]
  \frametitle{What is TDD?}
  \small

  The basic steps of TDD are roughly as follows --
  \begin{enumerate}
  \item<1-> Decide upon feature to implement and methodology of
    testing
  \item<2-> Write tests for feature decided upon
  \item<3-> Just write enough code, so that the test can be run, but it fails.
  \item<4-> Improve the code, to just pass the test and at the same time
    passing all previous tests
  \item<5-> Run the tests to see, that all of them run successfully
  \item<6-> Refactor the code you've just written -- optimize the algorithm,
    remove duplication, add documentation, etc.
  \item<7-> Run the tests again, to see that all the tests still pass
  \item<8-> Go back to 1
  \end{enumerate}
\end{frame}

\section{Let us TDD}

\begin{frame}[fragile]
  \frametitle{First Test -- GCD}
  \begin{itemize}
  \item simple program -- GCD of two numbers
  \item What are our code units?
    \begin{itemize}
    \item Only one function \texttt{gcd}
    \item Takes two numbers as arguments
    \item Returns one number, which is their GCD
    \end{itemize}
\begin{lstlisting}
c = gcd(44, 23)
\end{lstlisting}
  \item c will contain the GCD of the two numbers.
  \end{itemize}
\end{frame}

\begin{frame}[fragile]
  \frametitle{Test Cases}
  \begin{itemize}
  \item Important to have test cases and expected outputs even before
    writing the first test!
  \item $a=48$, $b=48$, $GCD=48$
  \item $a=44$, $b=19$, $GCD=1$
  \item Tests are just a series of assertions
  \item True or False, depending on expected and actual behavior
  \end{itemize}

\end{frame}

\begin{frame}[fragile]
  \frametitle{Test Cases -- general idea}
  \small
\begin{lstlisting}
tc1 = gcd(48, 64)
if tc1 != 16:
    print "Failed for a=48, b=64. Expected 16. \
    Obtained %d instead." % tc1
    exit(1)

tc2 = gcd(44, 19)
if tc2 != 1:
    print "Failed for a=44, b=19. Expected 1. \
    Obtained %d instead." % tc2
    exit(1)

print "All tests passed!"
\end{lstlisting}
\begin{itemize}
\item The function \texttt{gcd} doesn't even exist!
\end{itemize}
\end{frame}

\begin{frame}[fragile]
  \frametitle{Test Cases -- code}
  \begin{itemize}
      \item Let us make it a function!
      \item Use assert!
  \end{itemize}
\end{frame}

\begin{frame}[fragile]
  \frametitle{Test Cases -- code}
\begin{lstlisting}
# gcd.py
def test_gcd():
    assert gcd(48, 64) == 16
    assert gcd(44, 19) == 1

test_gcd()
\end{lstlisting}
\end{frame}

\begin{frame}[fragile]
  \frametitle{Stubs}
  \begin{itemize}
  \item First write a very minimal definition of \texttt{gcd}
    \begin{lstlisting}
def gcd(a, b):
    pass
    \end{lstlisting}
  \item Written just, so that the tests can run
  \item Obviously, the tests are going to fail
  \end{itemize}
\end{frame}

\begin{frame}[fragile]
  \frametitle{\texttt{gcd.py}}
\begin{lstlisting}
def gcd(a, b):
    pass

def test_gcd():
    assert gcd(48, 64) == 16
    assert gcd(44, 19) == 1

if __name__ == '__main__':
    test_gcd()
\end{lstlisting}
\end{frame}

\begin{frame}[fragile]
  \frametitle{First run}
\begin{lstlisting}
$ python gcd.py
Traceback (most recent call last):
  File "gcd.py", line 9, in <module>
    test_gcd()
  File "gcd.py", line 5, in test_gcd
    assert gcd(48, 64) == 16
AssertionError
\end{lstlisting} %$

  \begin{itemize}
  \item We have our code unit stub, and a failing test.
  \item The next step is to write code, so that the test just passes.
  \end{itemize}
\end{frame}

\begin{frame}[fragile,plain]
  \frametitle{Euclidean Algorithm}
  \begin{itemize}
  \item Modify the \texttt{gcd} stub function
  \item Then, run the script to see if the tests pass.
  \end{itemize}
\begin{lstlisting}
def gcd(a, b):
    if a == 0:
        return b
    while b != 0:
        if a > b:
            a = a - b
        else:
            b = b - a
    return a
\end{lstlisting}
\begin{lstlisting}
$ python gcd.py
All tests passed!
\end{lstlisting} %$
  \begin{itemize}
  \item \alert{Success!}
  \end{itemize}
\end{frame}


\begin{frame}[fragile]
  \frametitle{Euclidean Algorithm -- Modulo}
  \begin{itemize}
  \item Repeated subtraction can be replaced by a modulo
  \item modulo of \texttt{a\%b} is always less than b
  \item when \texttt{a < b}, \texttt{a\%b} equals \texttt{a}
  \item Combine these two observations, and modify the code
\begin{lstlisting}
def gcd(a, b):
    while b != 0:
        a, b = b, a % b
    return a
\end{lstlisting}
  \item Check that the tests pass again
  \end{itemize}
\end{frame}

\begin{frame}[fragile]
  \frametitle{Euclidean Algorithm -- Recursive}
  \begin{itemize}
  \item Final improvement -- make \texttt{gcd} recursive
  \item More readable and easier to understand
\begin{lstlisting}
def gcd(a, b):
    if b == 0:
        return a
    return gcd(b, a%b)
\end{lstlisting}
  \item Check that the tests pass again
  \end{itemize}
\end{frame}

\begin{frame}[fragile]
  \frametitle{Document \texttt{gcd}}
  \begin{itemize}
  \item Undocumented function is as good as unusable
  \item Let's add a docstring \& We have our first test!
  \end{itemize}
\end{frame}
\begin{frame}[fragile]
  \frametitle{Document \texttt{gcd}}
\small
\begin{lstlisting}
def gcd(a, b):
    """Returns the Greatest Common Divisor of the
    two integers passed as arguments.

    Args:
      a: an integer
      b: another integer

    Returns: Greatest Common Divisor of a and b
    """
    if b == 0:
        return a
    return gcd(b, a%b)
\end{lstlisting}
\end{frame}


\begin{frame}[fragile]
  \frametitle{Persistent Test Cases}
  \begin{itemize}
  \item Tests should be pre-determined and written, before the code
  \item The file shall have multiple lines of test code/data
  \item Separates the code from the tests
  \end{itemize}
\end{frame}

\begin{frame}[fragile]
    \frametitle{Separate \texttt{test\_gcd.py}}
\begin{lstlisting}
from gcd import gcd

def test_gcd():
    assert gcd(48, 64) == 16
    assert gcd(44, 19) == 1

if __name__ == '__main__':
    test_gcd()
\end{lstlisting}
\end{frame}

\section{Using \texttt{pytest}}

\begin{frame}
  \frametitle{\texttt{pytest}}
  \begin{itemize}
  \item \url{doc.pytest.org}
  \item Is a powerful test runner!
  \item Detailed info from simple assert statements
  \item Automatic discovery of tests
  \item Modular fixures
  \end{itemize}
\end{frame}

\subsection{Basics}

\begin{frame}[fragile]
  \frametitle{Lets start using it}
  \begin{lstlisting}
    $ pytest
  \end{lstlisting}%$
  Note that it automatically tested \typ{test\_gcd.py}\\

  Now try:
  \begin{lstlisting}
    $ pytest -h
  \end{lstlisting}%$
\end{frame}

\begin{frame}[fragile]
  \frametitle{Lets start using it}
  Let us make a mistake in \typ{gcd.py}
  \begin{lstlisting}
    $ pytest
  \end{lstlisting}%$
  Note that it produces far nicer output
\end{frame}

\begin{frame}[fragile]
  \frametitle{Some basic features}
\small
  \begin{lstlisting}
import pytest

def test_casting_raises_value_error():
    with pytest.raises(ValueError):
        int('asd')

def test_floating_point():
    assert 0.1 + 0.2 == pytest.approx(0.3)
  \end{lstlisting}
\end{frame}

\begin{frame}[fragile]
  \frametitle{Basic features}
\small
  \begin{lstlisting}
class TestSomething(object):
    def test_something_has_hello(self):
        x = 'hello'
        y = 'hello world'
        assert x in y

    def test_something_else(self):
        a = [1,2,3]
        b = {1, 2, 3}
        assert a == b
\end{lstlisting}
\end{frame}

\begin{frame}
  \frametitle{Another example as exercise}
  \begin{block}{Problem}
  Find all git repositories recursively inside a root directory
\end{block}
\pause
\begin{itemize}
\item Simplify the problem
\item Make it testable
\end{itemize}
\end{frame}

\begin{frame}
  \frametitle{Some rules for writing tests}
  \begin{itemize}
  \item The test should be runnable by anyone (even by a computer), almost anywhere.
  \item Don't write anything in the current directory (use a temporary directory).
  \item Cleanup any files you create while testing.
  \item Make sure tests do not affect global state.
  \end{itemize}
\end{frame}

\begin{frame}
  \frametitle{How do we do this?}
  \begin{itemize}
  \item Create some test data in a temporary directory
  \item Test!
  \item Cleanup the test data
  \end{itemize}
\end{frame}


\subsection{Fixtures}

\begin{frame}
  \frametitle{Fixtures}
  \begin{itemize}
  \item What if you want to do something common before all tests?
  \item Typically called a \textbf{fixture}
  \end{itemize}
\end{frame}

\begin{frame}[fragile]
  \frametitle{pytest fixtures}
  \begin{lstlisting}
import pytest
@pytest.fixture
def my_fixture():
    print('Hello there')

def test_foo(my_fixture):
    print('runing test')

  \end{lstlisting}
\end{frame}

\begin{frame}[fragile]
  \frametitle{pytest fixture cleanup}
  \begin{lstlisting}
import pytest
@pytest.fixture
def fix1():
    print('Hello there')
    yield
    print('Good bye')

def test_foo(fix1):
    print('runing test')

  \end{lstlisting}
\end{frame}

\begin{frame}[fragile]
  \frametitle{pytest fixture with data}
  \begin{lstlisting}
@pytest.fixture
def some_data(my_fixture):
    print('some_data fixture')
    yield {'a': 1, 'b': 2}
    # Cleanup
    print('Good bye: fix2')


def test_some_data(some_data):
    print(some_data)
    print('Running test_fix2')
  \end{lstlisting}
\end{frame}



\begin{frame}[fragile]
  \frametitle{pytest fixtures can use other fixtures}
  \begin{lstlisting}
@pytest.fixture
def fix2(my_fixture):
    yield
    # Cleanup
    print('Good bye')


def test_fix2(fix2):
    print('Running test_fix2')
  \end{lstlisting}
\end{frame}

\begin{frame}[fragile]
  \frametitle{Another way to use fixtures}
  \begin{lstlisting}
@pytest.mark.usefixtures('fix1', 'fix2')
class TestClass:
    def test_method1(self):
         # ...
  \end{lstlisting}
\end{frame}

\begin{frame}
  \frametitle{Fixture scope}
  \begin{itemize}
  \item \typ{@pytest.fixture(scope='module')}
  \item Scope can be
    \begin{itemize}
    \item \typ{'function'}: the default, executes for each function
    \item \typ{'module'}
    \item \typ{'class'}
    \item \typ{'session'}: run only once!
    \end{itemize}
  \end{itemize}
\end{frame}

\subsection{Writing nicer tests}

\begin{frame}
  \frametitle{Some rules for writing tests}
  \begin{itemize}
  \item The test should be runnable by anyone (even by a computer), almost anywhere.
  \item Don't write anything in the current directory (use a temporary directory).
  \item Cleanup any files you create while testing.
  \item Make sure tests do not affect global state.
  \end{itemize}
\end{frame}

\begin{frame}
  \frametitle{Writing nicer tests}
  \begin{itemize}
  \item Names of test functions
  \item Use descriptive function names
  \item Intent matters
  \item Segregate the test code into the following

    \begin{itemize}
    \item Given: what is the context of the test?
    \item When: what action is taken to actually test the problem
    \item Then: what do we actually ensure.
    \end{itemize}

  \item \url{dannorth.net/introducing-bdd/}
  \item \url{martinfowler.com/bliki/GivenWhenThen.html}
  \end{itemize}
\end{frame}

\begin{frame}
  \frametitle{Improve previous exercise}
  \begin{block}{Problem}
  Find all git repositories recursively inside a root directory
\end{block}
\begin{itemize}
  \item Use all ideas discussed previously
\end{itemize}

\end{frame}


\section{Travis-CI}

\begin{frame}
  \frametitle{Continuous integration}
  \begin{itemize}
  \item Run tests automatically
  \item On every push
  \item On every branch
  \item On every pull-request
  \item Free for public repos usually
  \end{itemize}
\end{frame}

\begin{frame}
  \frametitle{Continuous integration: services}
  \begin{itemize}
  \item \url{travis-ci.org}
  \item \url{appveyor.com}
  \item \url{shippable.com}
  \item \url{codeship.com}
  \item Work with github, bitbucket and gitlab
  \end{itemize}
\end{frame}

\begin{frame}
  \frametitle{Let us get started}
  \begin{enumerate}
  \item Commit code to local repo
  \item Push to public github repo (say pytest-tutorial)
  \item Visit \url{travis-ci.org}
  \item Connect Travis to Github
  \end{enumerate}
\end{frame}

\begin{frame}[fragile]
  \frametitle{\texttt{.travis.yml}}
  Create a \typ{.travis.yml} in your repo
  \begin{lstlisting}
language: python

python:
  - 2.7

install:
  - pip install -U pytest

script:
  - pytest
\end{lstlisting}
\end{frame}

\begin{frame}
  \frametitle{Starting travis}
  \begin{itemize}
  \item Push the \typ{.travis.yml}
  \item Thats it!
  \item Can add other versions of Python etc.
  \item Useful to validate YAML: \url{www.yamllint.com/}
  \end{itemize}
\end{frame}


\section{Coverage}

\begin{frame}
  \frametitle{Coverage of tests}
  \begin{itemize}
    \item Assess the amount of code that is covered by testing
    \item \url{coverage.readthedocs.io}
    \item \typ{pip install coverage}
  \end{itemize}
\end{frame}

\begin{frame}[fragile]
  \frametitle{Using coverage}
  \begin{lstlisting}
$ coverage run -m pytest my_package
$ coverage report
$ coverage html
  \end{lstlisting}%$
\end{frame}

\begin{frame}[fragile]
  \frametitle{Narrowing the source to look at}
  \begin{lstlisting}
$ coverage run --source . \
     -m pytest my_package
$ coverage report
$ coverage html
  \end{lstlisting}%$
\end{frame}

\begin{frame}[fragile]
  \frametitle{Using a \texttt{.coveragerc}}
  Create a \typ{.coveragerc}
  \begin{lstlisting}
[run]
source = .
branch = True
omit = */tests/*
  \end{lstlisting}
\end{frame}


\section{Codecov}


\section{Some advanced features}


\end{document}
