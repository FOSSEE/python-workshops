%%%%%%%%%%%%%%%%%%%%%%%%%%%%%%%%%%%%%%%%%%%%%%%%%%%%%%%%%%%%%%%%%%%%%%%%%%%%%%%%
% Tutorial slides on Python.
%
% Author: FOSSEE
% Copyright (c) 2017, FOSSEE, IIT Bombay
%%%%%%%%%%%%%%%%%%%%%%%%%%%%%%%%%%%%%%%%%%%%%%%%%%%%%%%%%%%%%%%%%%%%%%%%%%%%%%%%

\documentclass[14pt,compress]{beamer}


% Modified from: generic-ornate-15min-45min.de.tex
\mode<presentation>
{
  \usetheme{Madrid}
  \useoutertheme{infolines}
  \setbeamercovered{invisible}
}

\usepackage[english]{babel}
\usepackage[latin1]{inputenc}
%\usepackage{times}
\usepackage[T1]{fontenc}

% Taken from Fernando's slides.
\usepackage{ae,aecompl}
\usepackage{mathpazo,courier,euler}
\usepackage[scaled=.95]{helvet}

\definecolor{darkgreen}{rgb}{0,0.5,0}

\usepackage{listings}
\lstset{language=Python,
    basicstyle=\ttfamily\bfseries,
    commentstyle=\color{red}\itshape,
  stringstyle=\color{darkgreen},
  showstringspaces=false,
  keywordstyle=\color{blue}\bfseries}

%%%%%%%%%%%%%%%%%%%%%%%%%%%%%%%%%%%%%%%%%%%%%%%%%%%%%%%%%%%%%%%%%%%%%%
% Macros
\setbeamercolor{emphbar}{bg=blue!20, fg=black}
\newcommand{\emphbar}[1]
{\begin{beamercolorbox}[rounded=true]{emphbar}
      {#1}
 \end{beamercolorbox}
}
\newcounter{time}
\setcounter{time}{0}
\newcommand{\inctime}[1]{\addtocounter{time}{#1}{\tiny \thetime\ m}}

\newcommand{\typ}[1]{\textbf{\texttt{{#1}}}}


\newcommand{\kwrd}[1]{ \texttt{\textbf{\color{blue}{#1}}}  }

%%% This is from Fernando's setup.
% \usepackage{color}
% \definecolor{orange}{cmyk}{0,0.4,0.8,0.2}
% % Use and configure listings package for nicely formatted code
% \usepackage{listings}
% \lstset{
%    language=Python,
%    basicstyle=\small\ttfamily,
%    commentstyle=\ttfamily\color{blue},
%    stringstyle=\ttfamily\color{orange},
%    showstringspaces=false,
%    breaklines=true,
%    postbreak = \space\dots
% }

%\pgfdeclareimage[height=0.75cm]{iitmlogo}{iitmlogo}
%\logo{\pgfuseimage{iitmlogo}}


%% Delete this, if you do not want the table of contents to pop up at
%% the beginning of each subsection:
\AtBeginSubsection[]
{
  \begin{frame}<beamer>
    \frametitle{Outline}
    \tableofcontents[currentsection,currentsubsection]
  \end{frame}
}

\AtBeginSection[]
{
  \begin{frame}<beamer>
    \frametitle{Outline}
    \tableofcontents[currentsection,currentsubsection]
  \end{frame}
}

% If you wish to uncover everything in a step-wise fashion, uncomment
% the following command:
%\beamerdefaultoverlayspecification{<+->}

%\includeonlyframes{current,current1,current2,current3,current4,current5,current6}


%%%%%%%%%%%%%%%%%%%%%%%%%%%%%%%%%%%%%%%%%%%%%%%%%%%%%%%%%%%%%%%%%%%%%%
% Title page
\title[Basic Python]{Practice exercises: data structures}

\author[FOSSEE Team] {The FOSSEE Group}

\institute[FOSSEE -- IITB] {Department of Aerospace Engineering\\IIT Bombay}
\date[] {Mumbai, India}
%%%%%%%%%%%%%%%%%%%%%%%%%%%%%%%%%%%%%%%%%%%%%%%%%%%%%%%%%%%%%%%%%%%%%%


%%%%%%%%%%%%%%%%%%%%%%%%%%%%%%%%%%%%%%%%%%%%%%%%%%%%%%%%%%%%%%%%%%%%%%
% DOCUMENT STARTS
\begin{document}

\begin{frame}
  \titlepage
\end{frame}

\begin{frame}[fragile,plain]
  \frametitle{Note: Python 2.x and 3.x}

 If you are using Python 2.x
  \begin{itemize}
  \item Use \typ{raw\_input} instead of \typ{input}
  \item Use the following for \typ{print}
  \end{itemize}
 \begin{lstlisting}
from __future__ import print_function
\end{lstlisting}
\end{frame}

\begin{frame}[plain]
  \frametitle{Exercise: simple list creation}
  \begin{enumerate}
  \item Ask the user to enter an integer, \typ{n}
  \item Create a list of integers from 0 to \typ{n-1}
  \item Print this list
  \end{enumerate}
\end{frame}

\begin{frame}[fragile,plain]
\frametitle{Solution}
\begin{lstlisting}
  n = int(input())
  x = list(range(n))
  print(x)
\end{lstlisting}
\end{frame}

\begin{frame}[plain]
  \frametitle{Exercise: more list creation}
  \begin{enumerate}
  \item Ask the user to enter an integer, \typ{n}
  \item Store the square of all the odd numbers less than \typ{n} in a list
  \item Print this list
  \end{enumerate}
\end{frame}

\begin{frame}[fragile,plain]
\frametitle{Solution}
\begin{lstlisting}
  n = int(input())
  result = []
  for i in range(1, n, 2):
      result.append(i*i)
  print(result)
\end{lstlisting}
\end{frame}

\begin{frame}[plain]
  \frametitle{Exercise: more list creation}
  \begin{enumerate}
  \item Ask the user to enter an integer, \typ{n}
  \item Store the square of all the odd numbers less than \typ{n}
  \item Print a tuple of this list
  \end{enumerate}
\end{frame}

\begin{frame}[fragile,plain]
\frametitle{Solution}
\begin{lstlisting}
  n = int(input())
  result = []
  for i in range(1, n, 2):
      result.append(i*i)
  print(tuple(result))
\end{lstlisting}
\end{frame}


\begin{frame}[plain,fragile]
  \frametitle{Hint: string to list/tuple}
  Here is an easy way to convert a string to a list of characters.

  Try this:
  \begin{lstlisting}
    In []: x = 'hello'
    In []: print(list(x))
    In []: print(tuple(x))
  \end{lstlisting}
\end{frame}

\begin{frame}[plain]
  \frametitle{Exercise: list of Fibonacci}
  \begin{enumerate}
  \item Ask the user to enter an integer, \typ{n} ($\geq 1$)
  \item Store the first \typ{n} numbers of the Fibonnaci series in a list
  \item Print this list
  \end{enumerate}
\end{frame}

\begin{frame}[fragile,plain]
\frametitle{Solution}
\begin{lstlisting}
  n = int(input())
  a, b = 0, 1
  result = [0]
  for i in range(n):
      result.append(a)
      a, b = b, a+b
  print(result)
\end{lstlisting}
\end{frame}

\begin{frame}[plain,fragile]
  \frametitle{Exercise: square a list of integers}
  \begin{enumerate}
  \item Ask the user to enter a list of integers separated by spaces
  \item Convert this into a list of integers but square each element
  \item Print this list
  \end{enumerate}
  For example, if the user enters \typ{1 2 3 4}, print:
  \begin{lstlisting}
    [1, 4, 9, 16]
  \end{lstlisting}
\end{frame}

\begin{frame}[fragile,plain]
\frametitle{Solution}
\begin{lstlisting}
  text = input()
  result = []
  for item in text.split():
      x = int(item)
      result.append(x*x)
  print(result)
\end{lstlisting}
\end{frame}

\begin{frame}[plain,fragile]
  \frametitle{Exercise: list of tuples}
  \begin{enumerate}
  \item Ask the user to enter a list of integers separated by spaces
  \item Convert this into a list of integers but square each element
  \item Store the integer and its square in a tuple, put this into a list
  \item Print this list
  \end{enumerate}
  For example, if the user enters \typ{1 2 3 4}, print:
  \begin{lstlisting}
    [(1, 1), (2, 4), (3, 9), (4, 16)]
  \end{lstlisting}
\end{frame}

\begin{frame}[fragile,plain]
\frametitle{Solution}
\begin{lstlisting}
  text = input()
  result = []
  for item in text.split():
      x = int(item)
      result.append((x, x*x))
  print(result)
\end{lstlisting}
\end{frame}

\begin{frame}[plain,fragile]
  \frametitle{Hint: iterating over a list of tuples}
  Consider the following code:
  \begin{small}
  \begin{lstlisting}
In []: data = [(1, 1), (2, 4), (3, 9), (4, 16)]
\end{lstlisting}
\end{small}
We can iterate over this as follows:
\begin{small}
\begin{lstlisting}
In []: for x, y in data:
.....:     print(x, y)
\end{lstlisting}
\end{small}
\end{frame}


\begin{frame}[plain,fragile]
  \frametitle{Exercise: list methods}
  \begin{enumerate}
  \item Ask the user to enter a string
  \item Convert this into a list of characters
  \item Sort this list in ascending order
  \item Now eliminate any repeated values in this list
  \item Print this list
  \end{enumerate}
  For example, if the user enters \typ{hello}, print:
  \begin{lstlisting}
    ['e', 'h', 'l', 'o']
  \end{lstlisting}
\end{frame}

\begin{frame}[fragile,plain]
\frametitle{Solution}
\begin{lstlisting}
  text = input()
  chars = list(text)
  chars.sort()
  result = []
  for c in chars:
      if c not in result:
          result.append(c)
  print(result)
\end{lstlisting}
\end{frame}

\begin{frame}[fragile,plain]
\frametitle{Another solution}
\begin{lstlisting}
  text = input()
  chars = set(text)
  chars = list(chars)
  chars.sort()
  print(chars)
\end{lstlisting}
\end{frame}

\begin{frame}[fragile,plain]
\frametitle{Another solution}
\begin{lstlisting}
  chars = set(input())
  print(sorted(chars))
\end{lstlisting}
\end{frame}

\begin{frame}[plain,fragile]
  \frametitle{Exercise: dictionaries}
  \begin{enumerate}
  \item Ask the user to enter a string
  \item Convert this to lower case
  \item Count the number of occurrences of each character in the string
  \item Hint: use a dict
  \item Print the result in sorted order of the characters
  \end{enumerate}
  For example, if the user enters \typ{helloo}, print:
  \begin{lstlisting}
    e 1
    h 1
    l 2
    o 2
  \end{lstlisting}
\end{frame}

\begin{frame}[fragile,plain]
\frametitle{Solution}
\begin{lstlisting}
  text = input().lower()
  result = {}
  for char in text:
      if char in result:
          result[char] += 1
      else:
          result[char] = 1

  for char in sorted(result):
      print(char, result[char])
\end{lstlisting}
\end{frame}

\begin{frame}[plain]
  \frametitle{Problem: Day of the Week}
  \begin{block}{Problem}
    Given a list, \texttt{week}, containing names of the days of the
    week and a string \texttt{s}, check if the string is a day of the
    week. We should be able to check for any of the forms like,
    \emph{sat, Sat, SAT}
  \end{block}
  \begin{itemize}
  \item Ask the user for a string of length 3
  \item Check if the string is a day of the week
  \item Print True or False
  \end{itemize}
\end{frame}

\begin{frame}[fragile, plain]
  \frametitle{Solution}
  \small
  \begin{lstlisting}
week = 'mon tue wed thu fri sat sun'.split()
s = input().lower()
print(s in week)
\end{lstlisting}
\end{frame}


\begin{frame}[plain]
  {Problem: datestring to date tuple}

  You are given date strings of the form ``29 Jul, 2009'', or ``4 January
  2008''. In other words a number, a string and another number, with a comma
  sometimes separating the items.

  Write a program that takes such a string as input and prints a tuple (yyyy,
  mm, dd) where all three elements are ints.
\end{frame}

\begin{frame}[fragile]
  \frametitle{Solution}
 \small
  \begin{lstlisting}
months = ('jan feb mar apr may jun jul ' +
         'aug sep oct nov dec').split()
month2mm = {}
for i in range(len(months)):
    month2mm[months[i]] = i

date = input()
date = date.replace(',', ' ')
day, month, year = date.split()

dd, yyyy = int(day), int(year)
mon = month[:3].lower()
mm = month2mm[mon]
print((yyyy, mm, dd))
\end{lstlisting}
\end{frame}
\end{document}
