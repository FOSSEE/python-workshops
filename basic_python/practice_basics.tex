%%%%%%%%%%%%%%%%%%%%%%%%%%%%%%%%%%%%%%%%%%%%%%%%%%%%%%%%%%%%%%%%%%%%%%%%%%%%%%%%
% Tutorial slides on Python.
%
% Author: FOSSEE
% Copyright (c) 2009-2017, FOSSEE, IIT Bombay
%%%%%%%%%%%%%%%%%%%%%%%%%%%%%%%%%%%%%%%%%%%%%%%%%%%%%%%%%%%%%%%%%%%%%%%%%%%%%%%%

\documentclass[14pt,compress]{beamer}


% Modified from: generic-ornate-15min-45min.de.tex
\mode<presentation>
{
  \usetheme{Warsaw}
  \useoutertheme{infolines}
  \setbeamercovered{invisible}
}

\usepackage[english]{babel}
\usepackage[latin1]{inputenc}
%\usepackage{times}
\usepackage[T1]{fontenc}

% Taken from Fernando's slides.
\usepackage{ae,aecompl}
\usepackage{mathpazo,courier,euler}
\usepackage[scaled=.95]{helvet}

\definecolor{darkgreen}{rgb}{0,0.5,0}

\usepackage{listings}
\lstset{language=Python,
    basicstyle=\ttfamily\bfseries,
    commentstyle=\color{red}\itshape,
  stringstyle=\color{darkgreen},
  showstringspaces=false,
  keywordstyle=\color{blue}\bfseries}

%%%%%%%%%%%%%%%%%%%%%%%%%%%%%%%%%%%%%%%%%%%%%%%%%%%%%%%%%%%%%%%%%%%%%%
% Macros
\setbeamercolor{emphbar}{bg=blue!20, fg=black}
\newcommand{\emphbar}[1]
{\begin{beamercolorbox}[rounded=true]{emphbar}
      {#1}
 \end{beamercolorbox}
}
\newcounter{time}
\setcounter{time}{0}
\newcommand{\inctime}[1]{\addtocounter{time}{#1}{\tiny \thetime\ m}}

\newcommand{\typ}[1]{\textbf{\texttt{{#1}}}}


\newcommand{\kwrd}[1]{ \texttt{\textbf{\color{blue}{#1}}}  }

%%% This is from Fernando's setup.
% \usepackage{color}
% \definecolor{orange}{cmyk}{0,0.4,0.8,0.2}
% % Use and configure listings package for nicely formatted code
% \usepackage{listings}
% \lstset{
%    language=Python,
%    basicstyle=\small\ttfamily,
%    commentstyle=\ttfamily\color{blue},
%    stringstyle=\ttfamily\color{orange},
%    showstringspaces=false,
%    breaklines=true,
%    postbreak = \space\dots
% }

%\pgfdeclareimage[height=0.75cm]{iitmlogo}{iitmlogo}
%\logo{\pgfuseimage{iitmlogo}}


%% Delete this, if you do not want the table of contents to pop up at
%% the beginning of each subsection:
\AtBeginSubsection[]
{
  \begin{frame}<beamer>
    \frametitle{Outline}
    \tableofcontents[currentsection,currentsubsection]
  \end{frame}
}

\AtBeginSection[]
{
  \begin{frame}<beamer>
    \frametitle{Outline}
    \tableofcontents[currentsection,currentsubsection]
  \end{frame}
}

% If you wish to uncover everything in a step-wise fashion, uncomment
% the following command:
%\beamerdefaultoverlayspecification{<+->}

%\includeonlyframes{current,current1,current2,current3,current4,current5,current6}


%%%%%%%%%%%%%%%%%%%%%%%%%%%%%%%%%%%%%%%%%%%%%%%%%%%%%%%%%%%%%%%%%%%%%%
% Title page
\title[Basic Python]{Practice exercises: Basics}

\author[FOSSEE Team] {The FOSSEE Group}

\institute[FOSSEE -- IITB] {Department of Aerospace Engineering\\IIT Bombay}
\date[] {Mumbai, India}
%%%%%%%%%%%%%%%%%%%%%%%%%%%%%%%%%%%%%%%%%%%%%%%%%%%%%%%%%%%%%%%%%%%%%%


%%%%%%%%%%%%%%%%%%%%%%%%%%%%%%%%%%%%%%%%%%%%%%%%%%%%%%%%%%%%%%%%%%%%%%
% DOCUMENT STARTS
\begin{document}

\begin{frame}
  \titlepage
\end{frame}

\begin{frame}[fragile,plain]
  \frametitle{Note: Python 2.x and 3.x}

 If you are using Python 2.x
  \begin{itemize}
  \item Use \typ{raw\_input} instead of \typ{input}
  \item Use the following for \typ{print}
  \end{itemize}
 \begin{lstlisting}
from __future__ import print_function
\end{lstlisting}
\end{frame}

\begin{frame}[plain]
  \frametitle{Exercise: print string}
  \begin{enumerate}
  \item Ask the user to enter a name
    \begin{itemize}
    \item use \typ{input()} or \typ{raw\_input()}
    \end{itemize}
  \item Lets say the user entered: \\ \typ{abc}\\  then print \\ \typ{hello abc}
  \end{enumerate}
\end{frame}

\begin{frame}[fragile,plain]
\frametitle{Possible solution}
\begin{lstlisting}
  name = input() # Or raw_input()
  print("Hello", name)
\end{lstlisting}

\end{frame}


\begin{frame}[plain]
  \frametitle{Exercise: input prompt}
  \begin{enumerate}
  \item Ask the user to enter a name but give them a prompt:\\ \typ{"Please
    enter your name: "} \\ (note the trailing space)
  \item Lets say the user entered: \\ \typ{abc}\\  then print \\ \typ{hello abc}
  \end{enumerate}
\end{frame}

\begin{frame}[fragile,plain]
\frametitle{Possible solution}
\begin{lstlisting}
name = input("Please enter your name: ")
print("Hello", name)
\end{lstlisting}

\end{frame}

\begin{frame}[plain]
  \frametitle{Exercise: integers}
  \begin{enumerate}
  \item Ask the user for a single integer (no prompt string)
  \item Print the square of this number
  \end{enumerate}
\end{frame}

\begin{frame}[fragile,plain]
\frametitle{Possible solution}
\begin{lstlisting}
  x_str = input()
  x = int(x_str)
  print(x*x)
\end{lstlisting}

\end{frame}


\begin{frame}[plain]
  \frametitle{Exercise: digits of integer}
  \begin{enumerate}
  \item Ask the user for a single integer (use an empty prompt)
  \item Square this integer
  \item Print the number of digits the squared integer has
  \end{enumerate}
\end{frame}

\begin{frame}[fragile,plain]
\frametitle{Possible solution}
\begin{lstlisting}
  x_str = input()
  x = int(x_str)
  y_str = str(x*x)
  print(len(y_str))
\end{lstlisting}

\end{frame}

\begin{frame}[fragile,plain]
  \frametitle{Exercise: complex numbers}
  \begin{enumerate}
  \item Ask the user for a single complex number
  \item If the number entered was \typ{1+2j}, print the following:
    \begin{lstlisting}
      1 2
    \end{lstlisting}
  \item Print the absolute value of this complex number
  \item Print the conjugate of this complex number
  \end{enumerate}
\end{frame}

\begin{frame}[fragile,plain]
\frametitle{Possible solution}
\begin{lstlisting}
  z_str = input()
  z = complex(z_str)
  print(z.real, z.imag)
  print(abs(z))
  print(z.conjugate())
\end{lstlisting}

\end{frame}

\begin{frame}[plain]
  \frametitle{Exercise: Booleans}
  \begin{enumerate}
  \item Ask the user to enter an integer (use an empty prompt)
  \item Print \typ{True} if the number is odd print \typ{False} otherwise
  \end{enumerate}
\end{frame}

\begin{frame}[fragile,plain]
\frametitle{Possible solution}
\begin{lstlisting}
  x_str = input()
  x = int(x_str)
  print(x%2 == 1)
\end{lstlisting}
\end{frame}

\begin{frame}[plain]
  \frametitle{Exercise: Booleans}
  \begin{enumerate}
  \item Ask the user to enter an integer (use an empty prompt)
  \item Print \typ{True} if the number is even print \typ{False} otherwise
  \end{enumerate}
\end{frame}

\begin{frame}[fragile,plain]
\frametitle{Possible solution}
\begin{lstlisting}
  x_str = input()
  x = int(x_str)
  print(x%2 == 0)
\end{lstlisting}
\end{frame}

\begin{frame}[plain]
  \frametitle{Exercise: string operations 1}
  \begin{itemize}
  \item Ask the user to enter a name (use an empty prompt)
  \item Print the name entered but in upper case
  \item For example, if the user enters 'abc', print 'ABC'
  \end{itemize}
\end{frame}

\begin{frame}[fragile,plain]
\frametitle{Possible solution}
\begin{lstlisting}
  name = input()
  print("Hello", name.upper())
\end{lstlisting}
\end{frame}

\begin{frame}[plain]
  \frametitle{Exercise: string operations 2}
  \begin{enumerate}
  \item Ask the user to enter the name of a file (use an empty prompt)
  \item Add an extension '.txt' to the name and print it
  \item For example, if the user enters 'abc', print 'abc.txt'
  \end{enumerate}
\end{frame}

\begin{frame}[fragile,plain]
\frametitle{Possible solution}
\begin{lstlisting}
  name = input()
  print(name + '.txt')
\end{lstlisting}

\end{frame}


\begin{frame}[fragile,plain]
  \frametitle{Exercise: string slicing}
  \begin{enumerate}
  \item Ask the user to enter a string
  \item Print all parts of the string except the first character
  \item Print the string without the last character
  \item Then print the string in reverse
  \item Finally print every alternate character of the string starting from
    the first
  \end{enumerate}
  For example, if the user enters 'abcd', print the following:
  \begin{lstlisting}
    bcd
    abc
    dcba
    ac
  \end{lstlisting}
\end{frame}

\begin{frame}[fragile,plain]
\frametitle{Possible solution}
\begin{lstlisting}
  name = input()
  print(name[1:])
  print(name[:-1])
  print(name[::-1])
  print(name[::2])
\end{lstlisting}

\end{frame}

\begin{frame}[plain]
  \frametitle{Exercise: string manipulations}
  \begin{itemize}
  \item Ask the user to enter a string
  \item Count the number of vowels in the string
  \item The code should be case-insensitive
  \end{itemize}
\end{frame}

\begin{frame}[fragile,plain]
\frametitle{Possible solution}
\begin{lstlisting}
  x = input()
  x = x.lower()
  n = (x.count('a') + x.count('e') +
       x.count('i') + x.count('o') +
       x.count('u'))
  print(n)
\end{lstlisting}
\end{frame}

\begin{frame}[plain]
  \frametitle{Exercise: digits of an integer}
  Given a 2 digit integer\ \typ{x}, find the digits of the number.
  \vspace*{1em}

  \begin{itemize}
  \item For example, let us say \typ{x = 38}
  \item Find a way to get \typ{a = 3} and \typ{b = 8} using \typ{x}
  \item Print the digits, one in each line
  \end{itemize}
\end{frame}

\begin{frame}[fragile,plain]
  \frametitle{Possible Solution}
  \begin{lstlisting}
x = int(input())
a = x//10
b = x%10
print(a)
print(b)
\end{lstlisting}
\end{frame}

\begin{frame}[fragile,plain]
  \frametitle{Another Solution}
\begin{lstlisting}
  x = input()
  print(x[0])
  print(x[1])
\end{lstlisting}
\end{frame}

\end{document}
