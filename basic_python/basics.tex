%%%%%%%%%%%%%%%%%%%%%%%%%%%%%%%%%%%%%%%%%%%%%%%%%%%%%%%%%%%%%%%%%%%%%%%%%%%%%%%%
%Tutorial slides on Python.
%
% Author: FOSSEE
% Copyright (c) 2009-2017, FOSSEE, IIT Bombay
%%%%%%%%%%%%%%%%%%%%%%%%%%%%%%%%%%%%%%%%%%%%%%%%%%%%%%%%%%%%%%%%%%%%%%%%%%%%%%%%

\documentclass[14pt,compress]{beamer}


% Modified from: generic-ornate-15min-45min.de.tex
\mode<presentation>
{
  \usetheme{Warsaw}
  \useoutertheme{infolines}
  \setbeamercovered{invisible}
}

\usepackage[english]{babel}
\usepackage[latin1]{inputenc}
%\usepackage{times}
\usepackage[T1]{fontenc}

% Taken from Fernando's slides.
\usepackage{ae,aecompl}
\usepackage{mathpazo,courier,euler}
\usepackage[scaled=.95]{helvet}

\definecolor{darkgreen}{rgb}{0,0.5,0}

\usepackage{listings}
\lstset{language=Python,
    basicstyle=\ttfamily\bfseries,
    commentstyle=\color{red}\itshape,
  stringstyle=\color{darkgreen},
  showstringspaces=false,
  keywordstyle=\color{blue}\bfseries}

%%%%%%%%%%%%%%%%%%%%%%%%%%%%%%%%%%%%%%%%%%%%%%%%%%%%%%%%%%%%%%%%%%%%%%
% Macros
\setbeamercolor{emphbar}{bg=blue!20, fg=black}
\newcommand{\emphbar}[1]
{\begin{beamercolorbox}[rounded=true]{emphbar}
      {#1}
 \end{beamercolorbox}
}
\newcounter{time}
\setcounter{time}{0}
\newcommand{\inctime}[1]{\addtocounter{time}{#1}{\tiny \thetime\ m}}

\newcommand{\typ}[1]{\textbf{\texttt{{#1}}}}


\newcommand{\kwrd}[1]{ \texttt{\textbf{\color{blue}{#1}}}  }

%%% This is from Fernando's setup.
% \usepackage{color}
% \definecolor{orange}{cmyk}{0,0.4,0.8,0.2}
% % Use and configure listings package for nicely formatted code
% \usepackage{listings}
% \lstset{
%    language=Python,
%    basicstyle=\small\ttfamily,
%    commentstyle=\ttfamily\color{blue},
%    stringstyle=\ttfamily\color{orange},
%    showstringspaces=false,
%    breaklines=true,
%    postbreak = \space\dots
% }

%\pgfdeclareimage[height=0.75cm]{iitmlogo}{iitmlogo}
%\logo{\pgfuseimage{iitmlogo}}


%% Delete this, if you do not want the table of contents to pop up at
%% the beginning of each subsection:
\AtBeginSubsection[]
{
  \begin{frame}<beamer>
    \frametitle{Outline}
    \tableofcontents[currentsection,currentsubsection]
  \end{frame}
}

\AtBeginSection[]
{
  \begin{frame}<beamer>
    \frametitle{Outline}
    \tableofcontents[currentsection,currentsubsection]
  \end{frame}
}

% If you wish to uncover everything in a step-wise fashion, uncomment
% the following command:
%\beamerdefaultoverlayspecification{<+->}

%\includeonlyframes{current,current1,current2,current3,current4,current5,current6}


%%%%%%%%%%%%%%%%%%%%%%%%%%%%%%%%%%%%%%%%%%%%%%%%%%%%%%%%%%%%%%%%%%%%%%
% Title page
\title[Basic Python]{Python language: Basics}

\author[FOSSEE Team] {The FOSSEE Group}

\institute[FOSSEE -- IITB] {Department of Aerospace Engineering\\IIT Bombay}
\date[] {Mumbai, India}
%%%%%%%%%%%%%%%%%%%%%%%%%%%%%%%%%%%%%%%%%%%%%%%%%%%%%%%%%%%%%%%%%%%%%%


%%%%%%%%%%%%%%%%%%%%%%%%%%%%%%%%%%%%%%%%%%%%%%%%%%%%%%%%%%%%%%%%%%%%%%
% DOCUMENT STARTS
\begin{document}

\begin{frame}
  \titlepage
\end{frame}

\begin{frame}
  \frametitle{Outline}
  \tableofcontents
  % You might wish to add the option [pausesections]
\end{frame}

\section{Data types}

\begin{frame}
  \frametitle{Primitive Data types}
  \begin{itemize}
    \item Numbers: float, int, complex
    \item Strings
    \item Booleans
  \end{itemize}
\end{frame}

\subsection{Numbers}
\begin{frame}[fragile]
  \frametitle{Numbers}
  \begin{itemize}
    \item \kwrd{int}\\ whole number, no matter what the size!
  \begin{lstlisting}
In []: a = 13

In []: b = 99999999999999999999
  \end{lstlisting}
    \item \kwrd{float}
  \begin{lstlisting}
In []: p = 3.141592
  \end{lstlisting}
    \item \kwrd{complex}
  \begin{lstlisting}
In []: c = 3+4j
  \end{lstlisting}
  \end{itemize}
\end{frame}

\subsection{Booleans}
\begin{frame}[fragile]
  \frametitle{Booleans}
  \begin{lstlisting}
In []: t = True

In []: F = not t

In []: F or t
Out[]: True

In []: F and t
Out[]: False
  \end{lstlisting}
%%  \inctime{5}
\end{frame}

\begin{frame}[fragile]
  \frametitle{( )  for precedence}
  \begin{lstlisting}
In []: a = False
In []: b = True
In []: c = True

In []: (a and b) or c
Out[]: True

In []: a and (b or c)
Out[]: False
  \end{lstlisting}
%%  \inctime{5}
\end{frame}

\subsection{Strings}

\begin{frame}[fragile]
\frametitle{Strings}
Anything within ``quotes'' is a string!
\begin{lstlisting}
' This is a string '
" This too! "
""" This one too! """
''' And one more! '''
\end{lstlisting}
\end{frame}

\begin{frame}[fragile]
\frametitle{Strings}
Why so many?
\begin{lstlisting}
' "Do or do not.  No try." said Yoda.'
" ' is a mighty lonely quote."
\end{lstlisting}
The triple quoted ones can span multiple lines!

\begin{lstlisting}
""" The quick brown
fox jumped over
    the lazy dingbat.
"""
\end{lstlisting}
\end{frame}

\begin{frame}[fragile]
  \frametitle{Strings}
  \begin{lstlisting}
In []: w = "hello"

In []: print(w[0], w[1], w[-1])

In []: len(w)
Out[]: 5
  \end{lstlisting}
\end{frame}

\begin{frame}[fragile]
  \frametitle{Strings \ldots}
  \emphbar{Strings are immutable}
  \begin{lstlisting}
In []: w[0] = 'H'
  \end{lstlisting}
  \pause
  \begin{lstlisting}
--------------------------------------------
TypeError  Traceback (most recent call last)

<ipython console> in <module>()

TypeError: 'str' object does not
         support item assignment
  \end{lstlisting}
\end{frame}

\begin{frame}[fragile]
  \frametitle{Finding the type}
\begin{lstlisting}
In []: a = 1.0
In []: type(a)
Out[]: float
In []: type(1)
Out[]: int

In []: type(1+1j)
Out[]: complex

In []: type('hello')
Out[]: str
\end{lstlisting}
\end{frame}

\section{Operators}

\begin{frame}[fragile]
  \frametitle{Arithmetic operators}
  \small
  \begin{lstlisting}
In []: 1786 % 12
Out[]: 10

In []: 45 % 2
Out[]: 1

In []: 864675 % 10
Out[]: 5

In []: 3124 * 126789
Out[]: 396088836

In []: big = 1234567891234567890 ** 3

In []: verybig = big * big * big * big
  \end{lstlisting}
\end{frame}

\begin{frame}[fragile]
  \frametitle{Arithmetic operators}
  \begin{lstlisting}
In []: 17 / 2
Out[]: 8

In []: 17 / 2.0
Out[]: 8.5

In []: 17.0 / 2
Out[]: 8.5

In []: 17.0 / 8.5
Out[]: 2.0
  \end{lstlisting}
\end{frame}

\begin{frame}[fragile]
  \frametitle{Arithmetic operators: floor division}
  \begin{lstlisting}
In []: 17 // 2
Out[]: 8

In []: 17 // 2.0
Out[]: 8.0

In []: 17.0 // 2.0
Out[]: 8.0

In []: 17.0 // 8.6
Out[]: 1.0
  \end{lstlisting}
\end{frame}


\begin{frame}[fragile]
\frametitle{Arithmetic operators}
  \begin{lstlisting}
In []: c = 3+4j

In []: abs(c)
Out[]: 5.0

In []: c.imag
Out[]: 4.0

In []: c.real
Out[]: 3.0
  \end{lstlisting}
\end{frame}

\begin{frame}[fragile]
  \frametitle{Arithmetic operators}
  \begin{lstlisting}
In []: a = 7546

In []: a += 1
In []: a
Out[]: 7547

In []: a -= 5
In []: a

In []: a *= 2

In []: a /= 5
  \end{lstlisting}
\end{frame}


\begin{frame}[fragile]
  \frametitle{String operations}
  \begin{lstlisting}
In []: s = 'Hello'

In []: p = 'World'

In []: s + p
Out[]: 'HelloWorld'

In []: s * 4
Out[]: 'HelloHelloHelloHello'
  \end{lstlisting}
\end{frame}

\begin{frame}[fragile]
  \frametitle{String operations \ldots}
  \begin{lstlisting}
In []: s * s
  \end{lstlisting}
  \pause
  \begin{lstlisting}
--------------------------------------------
TypeError  Traceback (most recent call last)

<ipython console> in <module>()

TypeError: can`t multiply sequence by
                non-int of type `str`
  \end{lstlisting}
\end{frame}

\begin{frame}[fragile]
  \frametitle{String methods}
  \begin{lstlisting}
In []: a = 'Hello World'
In []: a.startswith('Hell')
Out[]: True

In []: a.endswith('ld')
Out[]: True

In []: a.upper()
Out[]: 'HELLO WORLD'

In []: a.lower()
Out[]: 'hello world'
  \end{lstlisting}
\end{frame}

\begin{frame}[fragile]
  \frametitle{String methods}
  \begin{lstlisting}
In []: a = '  Hello World  '
In []: b = a.strip()
In []: b
Out[]: 'Hello World'

In []: b.index('ll')
Out[]: 2
In []: b.replace('Hello', 'Goodbye')
Out[]: 'Goodbye World'
\end{lstlisting}
\end{frame}


\begin{frame}[fragile]
\frametitle{Strings: \typ{split} \& \typ{join}}
  \begin{lstlisting}
In []: chars = 'a b c'
In []: chars.split()
Out[]: ['a', 'b', 'c']
In []: ' '.join(['a', 'b', 'c'])
Out[]: 'a b c'
  \end{lstlisting}
  \begin{lstlisting}
In []: alpha = ', '.join(['a', 'b', 'c'])
In []: alpha
Out[]: 'a, b, c'
In []: alpha.split(', ')
Out[]: ['a', 'b', 'c']
  \end{lstlisting}
\end{frame}

\begin{frame}[fragile]
\frametitle{String formatting}
  \begin{lstlisting}
In []: x, y = 1, 1.234

In []: 'x is %s, y is %s' %(x, y)
Out[]: 'x is 1, y is 1.234'
  \end{lstlisting}
  \begin{itemize}
    \item \emph{\%d}, \emph{\%f} etc. available
  \end{itemize}
  \emphbar{\url{http://docs.python.org/library/stdtypes.html}}
%%  \inctime{10}
\end{frame}

\begin{frame}[fragile]
  \frametitle{Relational and logical operators}
  \begin{lstlisting}
In []: p, z, n = 1, 0, -1
In []: p == n
Out[]: False

In []: p >= n
Out[]: True

In []: n < z < p
Out[]: True

In []: p + n != z
Out[]: False
  \end{lstlisting}
\end{frame}

\begin{frame}[fragile]
  \frametitle{The \typ{assert} statement}
  \begin{itemize}
  \item You will see it in tests and your exam!
  \end{itemize}

  \begin{small}
\begin{lstlisting}
In []: assert p != n
In []: assert p == n
------------------------------------------------------
AssertionError      Traceback (most recent call last)
----> 1 assert p == n

AssertionError:
\end{lstlisting}
  \end{small}
  \begin{itemize}
  \item No error if condition is True
  \item Raises error if False
  \end{itemize}
\end{frame}

\begin{frame}[fragile]
  \frametitle{\typ{assert} examples}
  \begin{small}
\begin{lstlisting}
In []: assert p == n, "Oops condition failed"
------------------------------------------------------
AssertionError      Traceback (most recent call last)
----> 1 assert p == n

AssertionError: Oops condition failed
\end{lstlisting}
  \end{small}
  \begin{itemize}
  \item Can supply an optional message
  \end{itemize}
\end{frame}


\begin{frame}[fragile]
\frametitle{String containership}
  \begin{lstlisting}
In []: fruits = 'apple, banana, pear'
In []: 'apple' in fruits
Out[]: True

In []: 'potato' in fruits
Out[]: False

\end{lstlisting}
  \begin{itemize}
  \item Use tab complete to list other string methods
  \item Use \typ{?} to find more information
  \end{itemize}

\end{frame}

\begin{frame}[fragile]
  \frametitle{Built-ins}
  \begin{lstlisting}
In []: int(17 / 2.0)
Out[]: 8

In []: float(17 / 2)
Out[]: 8.0

In []: str(17 / 2.0)
Out[]: '8.5'

In []: round( 7.5 )
Out[]: 8.0
  \end{lstlisting}
\end{frame}

\begin{frame}[fragile]
  \frametitle{Odds and ends}
  \begin{itemize}
    \item Case sensitive
    \item Dynamically typed $\Rightarrow$ need not specify a type
      \begin{lstlisting}
In []: a = 1
In []: a = 1.1
In []: a = "Now I am a string!"
      \end{lstlisting}
    \item Comments:
      \begin{lstlisting}
In []: a = 1  # In-line comments
In []: # A comment line.
In []: a = "# Not a comment!"
      \end{lstlisting}
  \end{itemize}
%%  \inctime{15}
\end{frame}

\begin{frame}
  \frametitle{Exercise 1}
  Given a 2 digit integer\ \typ{x}, find the digits of the number.
\end{frame}

\begin{frame}[fragile]
  \frametitle{Possible Solution}
\begin{lstlisting}
In []: a = x//10
In []: b = x%10
In []: a*10 + b == x
\end{lstlisting}
\end{frame}

\begin{frame}
  \frametitle{Exercise 2}
  Given an arbitrary integer, count the number of digits it has.
\end{frame}

\begin{frame}[fragile]
\frametitle{Possible solution}
\begin{lstlisting}
In []: x = 12345678
In []: len(str(x))
\end{lstlisting}

  Sneaky solution!
\end{frame}



\section{Simple IO}
\begin{frame}[fragile]
  \frametitle{Simple IO: Console Input}
  \small
  \begin{itemize}
    \item \typ{input()} waits for user input.
      \begin{lstlisting}
In []: a = input()
5

In []: a
Out[]: '5'

In []: a = input('Enter a value: ')
Enter a value: 5
      \end{lstlisting}
    \item Prompt string is optional.
    \item All keystrokes are strings!
    \item \typ{int()} converts string to int.
  \end{itemize}
\end{frame}

\begin{frame}[fragile]
  \frametitle{Simple IO: Python 2.x vs 3.x}
  \begin{itemize}
  \item \typ{print} is familiar to us
  \item Changed from Python 2.x to 3.x
  \item We use the Python 3 convention here
    \vspace*{2em}
  \item If on Python 2.x do this:
  {
    \small
\begin{lstlisting}
In []: from __future__ import print_function
\end{lstlisting}
  }
  \item Safe to use in Python 3.x also
  \end{itemize}
\end{frame}

\begin{frame}[fragile]
  \frametitle{Simple IO: Console output}
  \begin{itemize}
  \item Put the following code snippet in a file \typ{hello1.py}
  \end{itemize}
  \begin{lstlisting}
print("Hello", "World")
print("Goodbye", "World")
\end{lstlisting}
Now run it like so:
\begin{lstlisting}
In []: %run hello1.py
Hello World
Goodbye World
  \end{lstlisting}
\end{frame}

\begin{frame}[fragile]
  \frametitle{Simple IO: Console output \ldots}
Put the following code snippet in a file \typ{hello2.py}
  \begin{lstlisting}
print("Hello", end=' ')
print("World")
\end{lstlisting}
Now run it like so:
\begin{lstlisting}
In []: %run hello2.py
Hello World
  \end{lstlisting}

\pause

\begin{block}{Mini Exercise}
  \begin{itemize}
  \item Read docs for \typ{print}
  \item Remember: use \typ{print?} \
  \end{itemize}
\end{block}
\end{frame}

\begin{frame}[fragile]
  \frametitle{What did we learn?}
  \begin{itemize}
    \item Data types: int, float, complex, boolean, string
    \item Operators: +, -, *, /, \%, **, +=, -=, *=, /=, >, <, <=, >=, ==, !=, a < b < c
    \item Simple IO: \kwrd{input} and \kwrd{print}
  \end{itemize}
\end{frame}

\begin{frame}
  \frametitle{Homework}
  \begin{itemize}
  \item Explore the various string methods
  \item Read the documentation for the string methods
  \item Explore the different builtins seen so far also
  \end{itemize}
\end{frame}


\end{document}

%% Questions for Quiz %%
%% ------------------ %%


\begin{frame}
\frametitle{\incqno }
  What is the largest integer value that can be represented natively by Python?
\end{frame}

\begin{frame}
\frametitle{\incqno }
  What is the result of 17.0 / 2?
\end{frame}

\begin{frame}
\frametitle{\incqno }
  Which of the following is not a type in Python?
  \begin{enumerate}
    \item int
    \item float
    \item char
    \item string
  \end{enumerate}
\end{frame}

\begin{frame}
\frametitle{\incqno }
How do you create a complex number with real part 2 and imaginary part
0.5.
\end{frame}

\begin{frame}
\frametitle{\incqno }
  What is the difference between \kwrd{print} \emph{x} and \kwrd{print} \emph{x,} ?
\end{frame}

\begin{frame}
\frametitle{\incqno }
  What does '*' * 40 produce?
\end{frame}

\begin{frame}[fragile]
\frametitle{\incqno }
    What is the output of:
    \begin{lstlisting}
In []: ', '.join(['a', 'b', 'c'])
    \end{lstlisting}
\end{frame}


\begin{frame}[fragile]
    \frametitle{\incqno}
  \begin{lstlisting}
In []: 47 % 3
  \end{lstlisting}
  What is the output?
\end{frame}
