%%%%%%%%%%%%%%%%%%%%%%%%%%%%%%%%%%%%%%%%%%%%%%%%%%%%%%%%%%%%%%%%%%%%%%%%%%%%%%%%
% Tutorial slides on Python.
%
% Author: FOSSEE
% Copyright (c) 2017, FOSSEE, IIT Bombay
%%%%%%%%%%%%%%%%%%%%%%%%%%%%%%%%%%%%%%%%%%%%%%%%%%%%%%%%%%%%%%%%%%%%%%%%%%%%%%%%

\documentclass[14pt,compress]{beamer}


% Modified from: generic-ornate-15min-45min.de.tex
\mode<presentation>
{
  \usetheme{Warsaw}
  \useoutertheme{infolines}
  \setbeamercovered{invisible}
}

\usepackage[english]{babel}
\usepackage[latin1]{inputenc}
%\usepackage{times}
\usepackage[T1]{fontenc}

% Taken from Fernando's slides.
\usepackage{ae,aecompl}
\usepackage{mathpazo,courier,euler}
\usepackage[scaled=.95]{helvet}

\definecolor{darkgreen}{rgb}{0,0.5,0}

\usepackage{listings}
\lstset{language=Python,
    basicstyle=\ttfamily\bfseries,
    commentstyle=\color{red}\itshape,
  stringstyle=\color{darkgreen},
  showstringspaces=false,
  keywordstyle=\color{blue}\bfseries}

%%%%%%%%%%%%%%%%%%%%%%%%%%%%%%%%%%%%%%%%%%%%%%%%%%%%%%%%%%%%%%%%%%%%%%
% Macros
\setbeamercolor{emphbar}{bg=blue!20, fg=black}
\newcommand{\emphbar}[1]
{\begin{beamercolorbox}[rounded=true]{emphbar}
      {#1}
 \end{beamercolorbox}
}
\newcounter{time}
\setcounter{time}{0}
\newcommand{\inctime}[1]{\addtocounter{time}{#1}{\tiny \thetime\ m}}

\newcommand{\typ}[1]{\textbf{\texttt{{#1}}}}


\newcommand{\kwrd}[1]{ \texttt{\textbf{\color{blue}{#1}}}  }

%%% This is from Fernando's setup.
% \usepackage{color}
% \definecolor{orange}{cmyk}{0,0.4,0.8,0.2}
% % Use and configure listings package for nicely formatted code
% \usepackage{listings}
% \lstset{
%    language=Python,
%    basicstyle=\small\ttfamily,
%    commentstyle=\ttfamily\color{blue},
%    stringstyle=\ttfamily\color{orange},
%    showstringspaces=false,
%    breaklines=true,
%    postbreak = \space\dots
% }

%\pgfdeclareimage[height=0.75cm]{iitmlogo}{iitmlogo}
%\logo{\pgfuseimage{iitmlogo}}


%% Delete this, if you do not want the table of contents to pop up at
%% the beginning of each subsection:
\AtBeginSubsection[]
{
  \begin{frame}<beamer>
    \frametitle{Outline}
    \tableofcontents[currentsection,currentsubsection]
  \end{frame}
}

\AtBeginSection[]
{
  \begin{frame}<beamer>
    \frametitle{Outline}
    \tableofcontents[currentsection,currentsubsection]
  \end{frame}
}

% If you wish to uncover everything in a step-wise fashion, uncomment
% the following command:
%\beamerdefaultoverlayspecification{<+->}

%\includeonlyframes{current,current1,current2,current3,current4,current5,current6}


%%%%%%%%%%%%%%%%%%%%%%%%%%%%%%%%%%%%%%%%%%%%%%%%%%%%%%%%%%%%%%%%%%%%%%
% Title page
\title[Files/exceptions]{Practice exercises: files and exceptions}

\author[FOSSEE Team] {The FOSSEE Group}

\institute[FOSSEE -- IITB] {Department of Aerospace Engineering\\IIT Bombay}
\date[] {Mumbai, India}
%%%%%%%%%%%%%%%%%%%%%%%%%%%%%%%%%%%%%%%%%%%%%%%%%%%%%%%%%%%%%%%%%%%%%%

%%%%%%%%%%%%%%%%%%%%%%%%%%%%%%%%%%%%%%%%%%%%%%%%%%%%%%%%%%%%%%%%%%%%%%
% DOCUMENT STARTS
\begin{document}

\begin{frame}
  \titlepage
\end{frame}


\begin{frame}[plain]
  \frametitle{Exercise: reading from a file}
  \begin{enumerate}
  \item Define a function called \typ{largest} which takes a single argument
  \item The argument passed will be an opened file object
  \item Read the data in the file
  \item Assume that the data is separated by spaces and are all numbers
  \item Find the maximum value in the file
  \item Do not use \typ{loadtxt}!
  \end{enumerate}
\end{frame}

\begin{frame}[fragile,plain]
\frametitle{Solution}
\begin{lstlisting}
  def largest(f):
      data = []
      for line in f:
          for field in line.split():
              data.append(float(field))
      return max(data)

\end{lstlisting}
\end{frame}

\begin{frame}[fragile,plain]
\frametitle{Another solution}
\begin{lstlisting}
def largest(f):
    res = -1e20
    for line in f:
        for field in line.split():
            res = max(res, float(field))
    return res

\end{lstlisting}
\end{frame}


\begin{frame}[plain]
  \frametitle{Exercise: reading/writing files}
  \begin{enumerate}
  \item Read the \typ{pendulum.txt} file
  \item Print the second column alone to another file called \typ{col2.txt}
  \item Remember to add a newline
  \end{enumerate}
\end{frame}

\begin{frame}[fragile,plain]
\frametitle{Solution}
\begin{lstlisting}
  f = open('pendulum.txt')
  out = open('col2.txt', 'w')
  for line in f:
      fields = line.split()
      out.write(fields[1] + '\n')
  f.close()
  out.close()
\end{lstlisting}
\end{frame}

\begin{frame}[fragile,plain]
\frametitle{Another solution}
\begin{lstlisting}
  f = open('pendulum.txt')
  out = open('col2.txt', 'w')
  for line in f:
      fields = line.split()
      print(fields[1], file=out)
  f.close()
  out.close()
\end{lstlisting}
\end{frame}

\begin{frame}[plain, fragile]
  \frametitle{Exercise: simple exceptions}
  \begin{enumerate}
  \item Write a function called \typ{my\_sum}
  \item The function is passed a single string with terms separated by spaces
  \item The string contains both names and integer values in arbitrary order
  \item Find the sum of all the numbers in the string
  \end{enumerate}
  For example:
  \begin{lstlisting}
>>> my_sum('1 fox, 2 dogs and 3 jackals')
6
>>> my_sum('3 blind mice and 1 man')
4
  \end{lstlisting}
\end{frame}

\begin{frame}[fragile,plain]
\frametitle{Possible solution}
\begin{lstlisting}
  def my_sum(s):
      total = 0
      for word in s.split():
          try:
              total += int(word)
          except ValueError:
              pass
      return total
 \end{lstlisting}
\end{frame}

\begin{frame}[plain, fragile]
  \frametitle{Exercise: catching exceptions}
  \small
  \begin{enumerate}
  \item Write a function called \typ{safe\_run(f, x)}
  \item \typ{f} is a function and \typ{x} is a value
  \item \typ{f} can raise either \typ{ValueError} or \typ{TypeError}
  \item Your function should return \typ{'OK'} if no exception is raised
  \item Return \typ{'ValueError'} if \typ{ValueError} is raised
  \item Return \typ{'TypeError'} if \typ{TypeError} is raised
  \end{enumerate}
  For example:
  \begin{lstlisting}
    >>> safe_run(float, 'A')
    'ValueError'
    >>> def f(x): return x + 2
    >>> safe_run(f, '2')
    'TypeError'
    >>> safe_run(float, '2')
    'OK'
  \end{lstlisting}

\end{frame}

\begin{frame}[fragile,plain]
\frametitle{Possible solution}
\begin{lstlisting}
  def safe_run(f, x):
      try:
          f(x)
      except ValueError:
          return 'ValueError'
      except TypeError:
          return 'TypeError'
      else:
          return 'OK'

 \end{lstlisting}
\end{frame}

\begin{frame}[plain, fragile]
  \frametitle{Exercise: raising exceptions}
  \begin{enumerate}
  \item Write a function \typ{func} which takes a single integer argument \typ{x}
  \item If \typ{x} is not a positive integer raise a \typ{ValueError}
  \item If \typ{x} is not an integer type raise a \typ{TypeError}
  \end{enumerate}
\end{frame}

\begin{frame}[fragile,plain]
  \frametitle{Possible solution}
  \small
\begin{lstlisting}
  def func(x):
      if type(x) != int:
          raise TypeError('Expected int')
      elif x < 0:
          raise ValueError('Got negative int')

\end{lstlisting}
\end{frame}

\begin{frame}
  \frametitle{What next?}
  \begin{itemize}
  \item Only covered the very basics
  \item More advanced topics remain
  \item Read the official Python tutorial:
    \url{docs.python.org/tutorial/}
  \end{itemize}
\end{frame}


\begin{frame}
  \centering
  \Huge

  Thank you!
\end{frame}
\end{document}

\end{document}
