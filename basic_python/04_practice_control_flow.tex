%%%%%%%%%%%%%%%%%%%%%%%%%%%%%%%%%%%%%%%%%%%%%%%%%%%%%%%%%%%%%%%%%%%%%%%%%%%%%%%%
% Tutorial slides on Python.
%
% Author: FOSSEE
% Copyright (c) 2009-2017, FOSSEE, IIT Bombay
%%%%%%%%%%%%%%%%%%%%%%%%%%%%%%%%%%%%%%%%%%%%%%%%%%%%%%%%%%%%%%%%%%%%%%%%%%%%%%%%

\documentclass[14pt,compress]{beamer}


% Modified from: generic-ornate-15min-45min.de.tex
\mode<presentation>
{
  \usetheme{Warsaw}
  \useoutertheme{infolines}
  \setbeamercovered{invisible}
}

\usepackage[english]{babel}
\usepackage[latin1]{inputenc}
%\usepackage{times}
\usepackage[T1]{fontenc}

% Taken from Fernando's slides.
\usepackage{ae,aecompl}
\usepackage{mathpazo,courier,euler}
\usepackage[scaled=.95]{helvet}

\definecolor{darkgreen}{rgb}{0,0.5,0}

\usepackage{listings}
\lstset{language=Python,
    basicstyle=\ttfamily\bfseries,
    commentstyle=\color{red}\itshape,
  stringstyle=\color{darkgreen},
  showstringspaces=false,
  keywordstyle=\color{blue}\bfseries}

%%%%%%%%%%%%%%%%%%%%%%%%%%%%%%%%%%%%%%%%%%%%%%%%%%%%%%%%%%%%%%%%%%%%%%
% Macros
\setbeamercolor{emphbar}{bg=blue!20, fg=black}
\newcommand{\emphbar}[1]
{\begin{beamercolorbox}[rounded=true]{emphbar}
      {#1}
 \end{beamercolorbox}
}
\newcounter{time}
\setcounter{time}{0}
\newcommand{\inctime}[1]{\addtocounter{time}{#1}{\tiny \thetime\ m}}

\newcommand{\typ}[1]{\textbf{\texttt{{#1}}}}


\newcommand{\kwrd}[1]{ \texttt{\textbf{\color{blue}{#1}}}  }

%%% This is from Fernando's setup.
% \usepackage{color}
% \definecolor{orange}{cmyk}{0,0.4,0.8,0.2}
% % Use and configure listings package for nicely formatted code
% \usepackage{listings}
% \lstset{
%    language=Python,
%    basicstyle=\small\ttfamily,
%    commentstyle=\ttfamily\color{blue},
%    stringstyle=\ttfamily\color{orange},
%    showstringspaces=false,
%    breaklines=true,
%    postbreak = \space\dots
% }

%\pgfdeclareimage[height=0.75cm]{iitmlogo}{iitmlogo}
%\logo{\pgfuseimage{iitmlogo}}


%% Delete this, if you do not want the table of contents to pop up at
%% the beginning of each subsection:
\AtBeginSubsection[]
{
  \begin{frame}<beamer>
    \frametitle{Outline}
    \tableofcontents[currentsection,currentsubsection]
  \end{frame}
}

\AtBeginSection[]
{
  \begin{frame}<beamer>
    \frametitle{Outline}
    \tableofcontents[currentsection,currentsubsection]
  \end{frame}
}

% If you wish to uncover everything in a step-wise fashion, uncomment
% the following command:
%\beamerdefaultoverlayspecification{<+->}

%\includeonlyframes{current,current1,current2,current3,current4,current5,current6}


%%%%%%%%%%%%%%%%%%%%%%%%%%%%%%%%%%%%%%%%%%%%%%%%%%%%%%%%%%%%%%%%%%%%%%
% Title page
\title[Basic Python]{Practice exercises: Control flow}

\author[FOSSEE Team] {The FOSSEE Group}

\institute[FOSSEE -- IITB] {Department of Aerospace Engineering\\IIT Bombay}
\date[] {Mumbai, India}
%%%%%%%%%%%%%%%%%%%%%%%%%%%%%%%%%%%%%%%%%%%%%%%%%%%%%%%%%%%%%%%%%%%%%%


%%%%%%%%%%%%%%%%%%%%%%%%%%%%%%%%%%%%%%%%%%%%%%%%%%%%%%%%%%%%%%%%%%%%%%
% DOCUMENT STARTS
\begin{document}

\begin{frame}
  \titlepage
\end{frame}

\begin{frame}[fragile,plain]
  \frametitle{Note: Python 2.x and 3.x}

 If you are using Python 2.x
  \begin{itemize}
  \item Use \typ{raw\_input} instead of \typ{input}
  \item Use the following for \typ{print}
  \end{itemize}
 \begin{lstlisting}
from __future__ import print_function
\end{lstlisting}
\end{frame}

\begin{frame}[plain]
  \frametitle{Exercise: simple \typ{if}}
  \begin{enumerate}
  \item Ask the user to enter an integer
  \item If the number is positive, print "Positive"
  \item If the number is negative, say -10 print:\\ "Negative of 10"
  \item If the number is zero, print "Zero"
  \item Finally, print "Done"
  \end{enumerate}
\end{frame}

\begin{frame}[fragile,plain]
\frametitle{Solution}
\begin{lstlisting}
  x = int(input())
  if x < 0:
      print("Negative of", abs(x))
  elif x > 0:
      print("Positive")
  else:
      print("Zero")
  print("Done")
\end{lstlisting}
\end{frame}

\begin{frame}[plain]
  \frametitle{Exercise: simple \typ{if} 2}
  \begin{enumerate}
  \item Ask the user to enter two integers, one on each line say \typ{x} and \typ{y}
  \item If \typ{x} is larger than \typ{y} print, "first", store the value of \typ{x} in a
    variable \typ{highest}
  \item If \typ{y} is larger than \typ{x} print, "second", store the value of \typ{y} in a
    variable \typ{highest}
  \item If they are equal, print "same", store either as the \typ{highest}
  \item Finally, print the square of the \typ{highest}
  \end{enumerate}
\end{frame}

\begin{frame}[fragile,plain]
\frametitle{Solution}
\begin{lstlisting}
  x = int(input())
  y = int(input())
  if x > y:
      print("first")
      highest = x
  elif y > x:
      print("second")
      highest = y
  else:
      print("same")
      highest = x
  print(highest*highest)
\end{lstlisting}
\end{frame}


\begin{frame}[plain]
  \frametitle{Simple \typ{while} loop 1}
  \begin{enumerate}
  \item Use a while loop to print the numbers from 1 to 10
  \end{enumerate}
\end{frame}

\begin{frame}[fragile,plain]
\frametitle{Solution}
\begin{lstlisting}
  x = 1
  while x < 11:
      print(x)
      x += 1
\end{lstlisting}
\end{frame}

\begin{frame}[plain]
  \frametitle{Simple \typ{while} loop 2}
  \begin{enumerate}
  \item Use a while loop to print the odd numbers from 1 to 10
  \end{enumerate}
\end{frame}

\begin{frame}[fragile,plain]
\frametitle{Solution}
\begin{lstlisting}
  x = 1
  while x < 11:
      print(x)
      x += 2
\end{lstlisting}
\end{frame}

\begin{frame}[plain]
  \frametitle{Simple \typ{while} loop 2}
  \begin{enumerate}
  \item Use a while loop to generate the numbers from 1 to 10
  \item If a number divisible by 3 is found, print ***
  \item If a number divisible by 5 is found, print *****
  \item Otherwise just print the number
  \end{enumerate}
\end{frame}

\begin{frame}[fragile,plain]
\frametitle{Solution}
\begin{lstlisting}
  x = 1
  while x < 11:
      if x%3 == 0:
          print('***')
      elif x%5 == 0:
          print('*****')
      else:
          print(x)
      x += 1
\end{lstlisting}
\end{frame}

\begin{frame}[plain]
  \frametitle{Simple \typ{while} loop 3}
  \begin{enumerate}
  \item Use a while loop to generate the numbers from 10 to 1
  \item For each number, print that many * characters
  \item That is, if the number is 1 print *, if the number is 3, print *** etc.
  \end{enumerate}
\end{frame}

\begin{frame}[fragile,plain]
\frametitle{Solution}
\begin{lstlisting}
  x = 10
  while x > 0:
      print('*'*x)
      x -= 1
\end{lstlisting}
\end{frame}

\begin{frame}[plain,fragile]
  \frametitle{Simple \typ{while} loop 4}
  \begin{enumerate}
  \item Ask the user for an integer, \typ{n} (>=2)
  \item Use a while loop to print \typ{n} numbers uniformly spaced between 0 and 1
  \item Include both 0 and 1
  \end{enumerate}
  For example if the user entered 3, print
  \begin{lstlisting}
    0.0
    0.5
    1.0
  \end{lstlisting}

\end{frame}

\begin{frame}[fragile,plain]
\frametitle{Solution}
\begin{lstlisting}
  n = int(input())
  dx = 1.0/(n-1)
  x = 0.0
  while x < (1.0-dx/2):
      print(x)
      x += dx
  print(1.0)
\end{lstlisting}
\end{frame}

\begin{frame}[plain,fragile]
  \frametitle{Simple \typ{for} loop 1}
  \begin{enumerate}
  \item Ask the user for an integer
  \item Use a for loop to generate the multiplication table for the number (up
    to 10)
  \end{enumerate}
  If the user entered 5, print the following
  \begin{lstlisting}
    5 x 1 = 5
    5 x 2 = 10
    5 x 3 = 15
    ...
  \end{lstlisting}
\end{frame}

\begin{frame}[fragile,plain]
\frametitle{Solution}
\begin{lstlisting}
  n = int(input())
  for i in range(1, 11):
      print(n, 'x', i, '=', n*i)
\end{lstlisting}
\end{frame}

\begin{frame}[plain,fragile]
  \frametitle{\typ{for} loop Fibonacci}
  \begin{enumerate}
  \item Ask the user for an integer, \typ{n} (>=1)
  \item Print the first \typ{n} numbers of the Fibonacci sequence
  \end{enumerate}
\end{frame}

\begin{frame}[fragile,plain]
\frametitle{Solution}
\begin{lstlisting}
  n = int(input())
  a, b = 0, 1
  print(a)
  for i in range(n-1):
      print(b)
      a, b = b, a+b
\end{lstlisting}
\end{frame}


\begin{frame}[plain,fragile]
  \frametitle{\typ{for} loop over list 1}
  \begin{enumerate}
  \item Ask the user for an integer, \typ{n}
  \item Use the range to create numbers from 1 to \typ{n}
  \item Print each value in this in a separate line
  \end{enumerate}
  For example let us say user enters \typ{3}, print
  \begin{lstlisting}
    1
    2
    3
  \end{lstlisting}
\end{frame}

\begin{frame}[fragile,plain]
\frametitle{Solution}
\begin{lstlisting}
  n = int(input())
  for x in range(1, n+1):
      print(x)
\end{lstlisting}
\end{frame}


\begin{frame}[plain,fragile]
  \frametitle{\typ{for} loop over array}
  \begin{enumerate}
  \item Ask the user for an integer, \typ{n}
  \item Use numpy's \typ{linspace} to create n points from 1 to 2
  \item Print each value in this in a separate line
  \end{enumerate}
  For example let us say user enters \typ{3}, print
  \begin{lstlisting}
    1.0
    1.5
    2.0
  \end{lstlisting}
\end{frame}

\begin{frame}[fragile,plain]
\frametitle{Solution}
\begin{lstlisting}
  from numpy import linspace
  n = int(input())
  data = linspace(1, 2, n)
  for x in data:
      print(x)
\end{lstlisting}
\end{frame}


\begin{frame}[plain,fragile]
  \frametitle{\typ{for} loop over list 2}
  \begin{enumerate}
  \item Ask the user for an list of fruits separated by a comma
  \item Print each fruit in a separate line
  \end{enumerate}
  For example let us say user enters \typ{apple, pear}, print
  \begin{lstlisting}
    apple
    pear
  \end{lstlisting}
\end{frame}

\begin{frame}[fragile,plain]
\frametitle{Solution}
\begin{lstlisting}
  fruits = input()
  fruits = fruits.split(',')
  for fruit in fruits:
      print(fruit)
\end{lstlisting}
\end{frame}

\begin{frame}[plain,fragile]
  \frametitle{\typ{for} loop over string}
  \begin{enumerate}
  \item Ask the user for a string
  \item Print each character of the string using a for loop
  \end{enumerate}
  For example let us say user enters \typ{box}, print
  \begin{lstlisting}
    b
    o
    x
  \end{lstlisting}
\end{frame}

\begin{frame}[fragile,plain]
\frametitle{Solution}
\begin{lstlisting}
  s = input()
  for char in s:
      print(char)
\end{lstlisting}
\end{frame}

\begin{frame}[plain,fragile]
  \frametitle{Nested \typ{for} loops}
  \begin{enumerate}
  \item Let us say the user supplies an integer, \typ{n} (empty prompt)
  \item Print an n x n matrix where each entry is the sum of the row + column
  \end{enumerate}
  For example let us say user enters \typ{3}, print
  \begin{lstlisting}
    0 1 2
    1 2 3
    2 3 4
  \end{lstlisting}
\end{frame}

\begin{frame}[fragile,plain]
\frametitle{Solution}
\begin{lstlisting}
  n = int(input())
  for i in range(n):
      for j in range(n):
          print(i+j, end=' ')
      print()
\end{lstlisting}
\end{frame}


\begin{frame}[plain,fragile]
  \frametitle{Exercise: Fibonacci divisible by 4}
  \begin{enumerate}
  \item Find the first number in the Fibonacci sequence divisible by 4 but > 8
    and less than 500.
  \end{enumerate}
\end{frame}

\begin{frame}[fragile,plain]
\frametitle{Solution}
\begin{lstlisting}
a, b = 0, 1
while b < 500:
    if b % 4 == 0 and b > 8:
        print(b)
        break
    a, b = b, a+b
\end{lstlisting}
\end{frame}


\end{document}
