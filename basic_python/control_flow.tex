%%%%%%%%%%%%%%%%%%%%%%%%%%%%%%%%%%%%%%%%%%%%%%%%%%%%%%%%%%%%%%%%%%%%%%%%%%%%%%%%
%Tutorial slides on Python.
%
% Author: FOSSEE
% Copyright (c) 2009-2017, FOSSEE, IIT Bombay
%%%%%%%%%%%%%%%%%%%%%%%%%%%%%%%%%%%%%%%%%%%%%%%%%%%%%%%%%%%%%%%%%%%%%%%%%%%%%%%%

\documentclass[14pt,compress]{beamer}


% Modified from: generic-ornate-15min-45min.de.tex
\mode<presentation>
{
  \usetheme{Warsaw}
  \useoutertheme{infolines}
  \setbeamercovered{invisible}
}

\usepackage[english]{babel}
\usepackage[latin1]{inputenc}
%\usepackage{times}
\usepackage[T1]{fontenc}

% Taken from Fernando's slides.
\usepackage{ae,aecompl}
\usepackage{mathpazo,courier,euler}
\usepackage[scaled=.95]{helvet}

\definecolor{darkgreen}{rgb}{0,0.5,0}

\usepackage{listings}
\lstset{language=Python,
    basicstyle=\ttfamily\bfseries,
    commentstyle=\color{red}\itshape,
  stringstyle=\color{darkgreen},
  showstringspaces=false,
  keywordstyle=\color{blue}\bfseries}

%%%%%%%%%%%%%%%%%%%%%%%%%%%%%%%%%%%%%%%%%%%%%%%%%%%%%%%%%%%%%%%%%%%%%%
% Macros
\setbeamercolor{emphbar}{bg=blue!20, fg=black}
\newcommand{\emphbar}[1]
{\begin{beamercolorbox}[rounded=true]{emphbar}
      {#1}
 \end{beamercolorbox}
}
\newcounter{time}
\setcounter{time}{0}
\newcommand{\inctime}[1]{\addtocounter{time}{#1}{\tiny \thetime\ m}}

\newcommand{\typ}[1]{\textbf{\texttt{{#1}}}}


\newcommand{\kwrd}[1]{ \texttt{\textbf{\color{blue}{#1}}}  }

%%% This is from Fernando's setup.
% \usepackage{color}
% \definecolor{orange}{cmyk}{0,0.4,0.8,0.2}
% % Use and configure listings package for nicely formatted code
% \usepackage{listings}
% \lstset{
%    language=Python,
%    basicstyle=\small\ttfamily,
%    commentstyle=\ttfamily\color{blue},
%    stringstyle=\ttfamily\color{orange},
%    showstringspaces=false,
%    breaklines=true,
%    postbreak = \space\dots
% }

%\pgfdeclareimage[height=0.75cm]{iitmlogo}{iitmlogo}
%\logo{\pgfuseimage{iitmlogo}}


%% Delete this, if you do not want the table of contents to pop up at
%% the beginning of each subsection:
\AtBeginSubsection[]
{
  \begin{frame}<beamer>
    \frametitle{Outline}
    \tableofcontents[currentsection,currentsubsection]
  \end{frame}
}

\AtBeginSection[]
{
  \begin{frame}<beamer>
    \frametitle{Outline}
    \tableofcontents[currentsection,currentsubsection]
  \end{frame}
}

% If you wish to uncover everything in a step-wise fashion, uncomment
% the following command:
%\beamerdefaultoverlayspecification{<+->}

%\includeonlyframes{current,current1,current2,current3,current4,current5,current6}


%%%%%%%%%%%%%%%%%%%%%%%%%%%%%%%%%%%%%%%%%%%%%%%%%%%%%%%%%%%%%%%%%%%%%%
% Title page
\title[Control flow]{Python language: Control Flow}

\author[FOSSEE Team] {The FOSSEE Group}

\institute[FOSSEE -- IITB] {Department of Aerospace Engineering\\IIT Bombay}
\date[] {Mumbai, India}
%%%%%%%%%%%%%%%%%%%%%%%%%%%%%%%%%%%%%%%%%%%%%%%%%%%%%%%%%%%%%%%%%%%%%%


%%%%%%%%%%%%%%%%%%%%%%%%%%%%%%%%%%%%%%%%%%%%%%%%%%%%%%%%%%%%%%%%%%%%%%
% DOCUMENT STARTS
\begin{document}

\begin{frame}
  \titlepage
\end{frame}

\begin{frame}
  \frametitle{Outline}
  \tableofcontents
  % You might wish to add the option [pausesections]
\end{frame}

\section{Control flow}

\begin{frame}
  \frametitle{Control flow constructs}
  \begin{itemize}
  \item \kwrd{if/elif/else}: branching
  \item \kwrd{while}: looping
  \item \kwrd{for}: iterating
  \item \kwrd{break, continue}: modify loop
  \item \kwrd{pass}: syntactic filler
  \end{itemize}
\end{frame}

\subsection{Basic Conditional flow}
\begin{frame}[fragile]
  \frametitle{\typ{if...elif...else} example}
Type the following code in an editor \& save as \alert{ladder.py}
{  \small
\begin{lstlisting}
x = int(input("Enter an integer: "))
if x < 0:
    print('Be positive!')
elif x == 0:
    print('Zero')
elif x == 1:
    print('Single')
else:
    print('More')
\end{lstlisting}
}
  %% \inctime{10}
\pause
\begin{itemize}
\item Run in IPython: \typ{\%run ladder.py}
\item Run on terminal: \typ{python ladder.py}
\end{itemize}
\end{frame}

\begin{frame}[fragile]
  \frametitle{Ternary operator}
  \begin{itemize}
  \item \texttt{score\_str} is either \texttt{'AA'} or a string of one
    of the numbers in the range 0 to 100.
  \item We wish to convert the string to a number using \texttt{int}
  \item Convert it to 0, when it is \texttt{'AA'}
  \item \texttt{if-else} construct or the ternary operator
  \end{itemize}
  \begin{lstlisting}
In []: if score_str != 'AA':
.....:     score = int(score_str)
.....: else:
.....:     score = 0
  \end{lstlisting}
\end{frame}
\begin{frame}[fragile]
  \frametitle{Ternary operator}
  With the ternary operator you can do this:
  \small
  \begin{lstlisting}
In []: ss = score_str
In []: score = int(ss) if ss != 'AA' else 0
  \end{lstlisting}
\end{frame}


\section{Control flow}
\subsection{Basic Looping}

\begin{frame}
  \frametitle{\typ{while}: motivational problem}
\begin{block}{Example: Fibonacci series}
  Sum of previous two elements defines the next:
  \begin{center}
    $0,\ \ 1,\ \ 1,\ \ 2,\ \ 3,\ \ 5,\ \ 8,\ \ 13,\ \ 21,\ \ ...$
  \end{center}
\end{block}
\end{frame}

\begin{frame}
  \frametitle{How do you solve this?}
  \begin{itemize}
  \item Task: Give computer, instructions to solve this
    \vspace*{2em}
  \item How would you solve it?
  \item How would you tell someone to solve it?
    \vspace*{1em}
  \item Assume you are given the starting values, 0 and 1.
  \end{itemize}

\end{frame}

\begin{frame}[fragile]
  \frametitle{\typ{while}: Fibonacci}
\begin{block}{Example: Fibonacci series}
  Sum of previous two elements defines the next:
  \begin{center}
    $0,\ \ 1,\ \ 1,\ \ 2,\ \ 3,\ \ 5,\ \ 8,\ \ 13,\ \ 21,\ \ ...$
  \end{center}
\end{block}
\pause
\begin{enumerate}
\item Start with: \typ{a, b = 0, 1}
  \pause
\item Next element: \typ{next = a + b}
  \pause
\item Shift \typ{a, b} to next values
  \begin{itemize}
  \item \typ{a = b}
  \item \typ{b = next}
  \end{itemize}
  \pause
\item Repeat steps 2, 3 when \typ{b < 30}
\end{enumerate}
\end{frame}


\begin{frame}[fragile]
  \frametitle{\typ{while}}
  \begin{lstlisting}
In []: a, b = 0, 1
In []: while b < 30:
  ...:     print(b, end=' ')
  ...:     next = a + b
  ...:     a = b
  ...:     b = next
  ...:
  ...:
\end{lstlisting}
  \begin{itemize}
  \item Do this manually to check logic
  \item Note: Indentation determines scope
  \end{itemize}
\end{frame}

\begin{frame}[fragile]
  \frametitle{\typ{while}}
  We can eliminate the temporary \typ{next}:
  \begin{lstlisting}
In []: a, b = 0, 1
In []: while b < 30:
  ...:     print(b, end=' ')
  ...:     a, b = b, a + b
  ...:
  ...:
1 1 2 3 5 8 13 21
\end{lstlisting}
  Simple!
\end{frame}


\begin{frame}[fragile]
  \frametitle{\typ{for} \ldots \typ{range()}}
Example: print squares of first \typ{5} numbers
  \begin{lstlisting}
In []: for i in range(5):
 ....:     print(i, i * i)
 ....:
 ....:
0 0
1 1
2 4
3 9
4 16
\end{lstlisting}
\end{frame}

\begin{frame}[fragile]
\frametitle{\typ{range()}}
\kwrd{range([start,] stop[, step])}\\
\begin{itemize}
  \item \typ{range()} returns a sequence of integers
  \item The \typ{start} and the \typ{step} arguments are optional
  \item \typ{stop} is not included in the sequence
\end{itemize}
\vspace*{.5in}
\begin{block}{Documentation convention}
  \begin{itemize}
    \item \alert{Anything within \typ{[]} is optional}
    \item Nothing to do with Python
  \end{itemize}
\end{block}
\end{frame}

\begin{frame}[fragile]
  \frametitle{\typ{for} \ldots \typ{range()}}
Example: print squares of odd numbers from 3 to 9
  \begin{lstlisting}
In []: for i in range(3, 10, 2):
 ....:     print(i, i * i)
 ....:
 ....:
3 9
5 25
7 49
9 81
\end{lstlisting}
%% \inctime{5}
\end{frame}

\begin{frame}
  \frametitle{Exercise with \typ{for}}

Convert the Fibonnaci sequence example to use a \typ{for} loop with range.

\end{frame}

\begin{frame}[fragile]
  \frametitle{Solution}
\begin{lstlisting}
a, b = 0, 1
for i in range(10):
    print(b, end=' ')
    a, b = b, a + b
\end{lstlisting}
  \vspace*{2em}
Note that the \typ{while} loop is a more natural fit here
\end{frame}

\begin{frame}
  \frametitle{\typ{break}, \typ{continue}, and\  \typ{pass}}
  \begin{itemize}
  \item Use \typ{break} to break out of loop
  \item Use \typ{continue} to skip an iteration
  \item Use \typ{pass} as syntactic filler
  \end{itemize}
\end{frame}

\begin{frame}[fragile]
  \frametitle{\typ{break} example}
  Find first number in Fibonnaci sequence < 100 divisible by 4:
\begin{lstlisting}
a, b = 0, 1
while b < 500:
    if b % 4 == 0:
        print(b)
        break
    a, b = b, a + b

\end{lstlisting}
\end{frame}

\begin{frame}[fragile]
  \frametitle{\texttt{continue}}
  \begin{itemize}
  \item Skips execution of rest of the loop on current iteration
  \item Jumps to the end of this iteration
  \item Squares of all odd numbers below 10, not multiples of 3
  \end{itemize}
  \begin{lstlisting}
  In []: for n in range(1, 10, 2):
  .....:     if n%3 == 0:
  .....:         continue
  .....:     print(n*n)
  \end{lstlisting}
\end{frame}


\begin{frame}[fragile]
  \frametitle{\typ{pass} example}
Try this:
\begin{lstlisting}
for i in range(5):
    if i % 2 ==  0:
        pass
    else:
        print(i, 'is Odd')
\end{lstlisting}
\begin{itemize}
\item \typ{pass}: does nothing
\item Keep Python syntactically happy
\end{itemize}
Another example:
\begin{lstlisting}
while True:
    pass
\end{lstlisting}
\end{frame}



\section{Exercises}

\begin{frame}{Problem 1.1: \emph{Armstrong} numbers}
  Write a program that displays all three digit numbers that are equal to the sum of the cubes of their digits. That is, print numbers $abc$ that have the property $abc = a^3 + b^3 + c^3$\\
For example, $153 = 1^3 + 5^3 + 3^3$\\
\vspace*{0.2in}

\begin{block}{Hints}
  \begin{itemize}
  \item Break problem into easier pieces
  \item How would you solve the problem?
  \item Can you explain to someone else how to solve it?
  \end{itemize}
\end{block}
\end{frame}

\begin{frame}[fragile]
  \frametitle{Some hints}
  \begin{itemize}
  \item What are the possible three digit numbers?
  \item Can you split 153 into its respective digits, $a=1, b=5, c=3$?
  \item With $a, b, c$ can you test if it is an Armstrong number?
  \end{itemize}
\end{frame}

\begin{frame}[fragile]
  \frametitle{Solution: part 1}
\begin{lstlisting}
x = 153
a = x//100
b = (x%100)//10
c = x%10

(a**3 + b**3 + c**3) == x
\end{lstlisting}
\end{frame}

\begin{frame}[fragile]
  \frametitle{Solution: part 2}
\begin{lstlisting}
x = 100
while x < 1000:
    print(x)
    x += 1  # x = x + 1
\end{lstlisting}
\end{frame}

\begin{frame}[fragile]
  \frametitle{Solution}
\begin{lstlisting}
x = 100
while x < 1000:
    a = x//100
    b = (x%100)//10
    c = x%10
    if (a**3 + b**3 + c**3) == x:
        print(x)
    x += 1
\end{lstlisting}
\end{frame}


\begin{frame}{Problem 1.2: Collatz sequence}
\begin{enumerate}
  \item Start with an arbitrary (positive) integer.
  \item If the number is even, divide by 2; if the number is odd, multiply by 3 and add 1.
  \item Repeat the procedure with the new number.
  \item It appears that for all starting values there is a cycle of 4, 2, 1 at which the procedure loops.
\end{enumerate}
    Write a program that accepts the starting value and prints out the Collatz sequence.
%% \inctime{5}
\end{frame}


\begin{frame}[fragile]
  \frametitle{What did we learn?}
  \begin{itemize}
    \item Conditionals: \kwrd{if elif else}
    \item Looping: \kwrd{while} \& \kwrd{for}
    \item \typ{range}
    \item \kwrd{break, continue, pass}
    \item Solving simple problems
  \end{itemize}
\end{frame}

\end{document}
