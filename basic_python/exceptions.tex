%%%%%%%%%%%%%%%%%%%%%%%%%%%%%%%%%%%%%%%%%%%%%%%%%%%%%%%%%%%%%%%%%%%%%%%%%%%%%%%%
%Tutorial slides on Python.
%
% Author: FOSSEE
% Copyright (c) 2017, FOSSEE, IIT Bombay
%%%%%%%%%%%%%%%%%%%%%%%%%%%%%%%%%%%%%%%%%%%%%%%%%%%%%%%%%%%%%%%%%%%%%%%%%%%%%%%%

\documentclass[14pt,compress]{beamer}

% Modified from: generic-ornate-15min-45min.de.tex
\mode<presentation>
{
  \usetheme{Warsaw}
  \useoutertheme{infolines}
  \setbeamercovered{invisible}
}

\usepackage[english]{babel}
\usepackage[latin1]{inputenc}
%\usepackage{times}
\usepackage[T1]{fontenc}

% Taken from Fernando's slides.
\usepackage{ae,aecompl}
\usepackage{mathpazo,courier,euler}
\usepackage[scaled=.95]{helvet}

\definecolor{darkgreen}{rgb}{0,0.5,0}

\usepackage{listings}
\lstset{language=Python,
    basicstyle=\ttfamily\bfseries,
    commentstyle=\color{red}\itshape,
  stringstyle=\color{darkgreen},
  showstringspaces=false,
  keywordstyle=\color{blue}\bfseries}

%%%%%%%%%%%%%%%%%%%%%%%%%%%%%%%%%%%%%%%%%%%%%%%%%%%%%%%%%%%%%%%%%%%%%%
% Macros
\setbeamercolor{emphbar}{bg=blue!20, fg=black}
\newcommand{\emphbar}[1]
{\begin{beamercolorbox}[rounded=true]{emphbar}
      {#1}
 \end{beamercolorbox}
}
\newcounter{time}
\setcounter{time}{0}
\newcommand{\inctime}[1]{\addtocounter{time}{#1}{\tiny \thetime\ m}}

\newcommand{\typ}[1]{\textbf{\texttt{{#1}}}}


\newcommand{\kwrd}[1]{ \texttt{\textbf{\color{blue}{#1}}}  }

%%% This is from Fernando's setup.
% \usepackage{color}
% \definecolor{orange}{cmyk}{0,0.4,0.8,0.2}
% % Use and configure listings package for nicely formatted code
% \usepackage{listings}
% \lstset{
%    language=Python,
%    basicstyle=\small\ttfamily,
%    commentstyle=\ttfamily\color{blue},
%    stringstyle=\ttfamily\color{orange},
%    showstringspaces=false,
%    breaklines=true,
%    postbreak = \space\dots
% }

%\pgfdeclareimage[height=0.75cm]{iitmlogo}{iitmlogo}
%\logo{\pgfuseimage{iitmlogo}}


%% Delete this, if you do not want the table of contents to pop up at
%% the beginning of each subsection:
\AtBeginSubsection[]
{
  \begin{frame}<beamer>
    \frametitle{Outline}
    \tableofcontents[currentsection,currentsubsection]
  \end{frame}
}

\AtBeginSection[]
{
  \begin{frame}<beamer>
    \frametitle{Outline}
    \tableofcontents[currentsection,currentsubsection]
  \end{frame}
}

% If you wish to uncover everything in a step-wise fashion, uncomment
% the following command:
%\beamerdefaultoverlayspecification{<+->}

%\includeonlyframes{current,current1,current2,current3,current4,current5,current6}


%%%%%%%%%%%%%%%%%%%%%%%%%%%%%%%%%%%%%%%%%%%%%%%%%%%%%%%%%%%%%%%%%%%%%%
% Title page
\title[Exceptions]{Python language: exceptions}

\author[FOSSEE Team] {The FOSSEE Group}

\institute[IIT Bombay] {Department of Aerospace Engineering\\IIT Bombay}
\date[] {Mumbai, India}
%%%%%%%%%%%%%%%%%%%%%%%%%%%%%%%%%%%%%%%%%%%%%%%%%%%%%%%%%%%%%%%%%%%%%%

%%%%%%%%%%%%%%%%%%%%%%%%%%%%%%%%%%%%%%%%%%%%%%%%%%%%%%%%%%%%%%%%%%%%%%
% DOCUMENT STARTS
\begin{document}

\begin{frame}
  \titlepage
\end{frame}

\begin{frame}{Motivation}
    \begin{itemize}
        \item How do you signal errors to a user?
    \end{itemize}
\end{frame}

\begin{frame}
  \frametitle{Exceptions}
  \begin{itemize}
  \item Python's way of notifying you of errors
  \item Several standard exceptions: \texttt{SyntaxError}, \texttt{IOError}
    etc.
  \item Users can also \texttt{raise} errors
  \item Users can create their own exceptions
  \item Exceptions can be ``caught'' via \texttt{try/except} blocks
  \end{itemize}
\end{frame}

\begin{frame}[fragile]
 \frametitle{Exceptions: examples}
 \begin{lstlisting}
In []: while True print('Hello world')
 \end{lstlisting}
\pause
  \begin{lstlisting}
File "<stdin>", line 1, in ?
    while True print('Hello world')
                   ^
SyntaxError: invalid syntax
\end{lstlisting}
\end{frame}

\begin{frame}[fragile]
 \frametitle{Exceptions: examples}
 \begin{lstlisting}
In []: print(spam)
\end{lstlisting}
\pause
\begin{lstlisting}
Traceback (most recent call last):
  File "<stdin>", line 1, in <module>
NameError: name 'spam' is not defined
\end{lstlisting}
\end{frame}

\begin{frame}[fragile]
 \frametitle{Exceptions: examples}
 \begin{lstlisting}
In []: 1 / 0
\end{lstlisting}
\pause
\begin{lstlisting}
Traceback (most recent call last):
  File "<stdin>", line 1, in <module>
ZeroDivisionError: integer division
or modulo by zero
\end{lstlisting}
\end{frame}


\begin{frame}[fragile]
  \frametitle{Exceptions: examples}
\begin{lstlisting}
In []: '2' + 2
\end{lstlisting}
\pause
\begin{lstlisting}
Traceback (most recent call last):
  File "<stdin>", line 1, in ?
TypeError: cannot concatenate 'str' and 'int' objects
\end{lstlisting}
\end{frame}


\begin{frame}[fragile]
  \frametitle{Processing user input}
  \begin{lstlisting}
prompt = 'Enter a number(Q to quit): '
a = input(prompt)

num = int(a) if a != 'Q' else 0
  \end{lstlisting}
  \emphbar{What if the user enters some other alphabet?}
\end{frame}


\begin{frame}[fragile]
  \frametitle{Handling Exceptions}
  Python provides a \typ{try} and \typ{except} clause.
  \begin{lstlisting}
prompt = 'Enter a number(Q to quit): '
a = input(prompt)
try:
    num = int(a)
    print(num)
except:
    if a == 'Q':
        print("Exiting ...")
    else:
        print("Wrong input ...")
  \end{lstlisting}
\end{frame}

\begin{frame}[fragile]
  \frametitle{Handling Exceptions a little better}
  Use specific exceptions; avoid blanket except clauses
  \begin{lstlisting}
prompt = 'Enter a number(Q to quit): '
a = input(prompt)
try:
    num = int(a)
    print(num)
except ValueError:
    if a == 'Q':
        print("Exiting ...")
    else:
        print("Wrong input ...")
  \end{lstlisting}
\end{frame}


\begin{frame}[fragile]
  \frametitle{Exceptions: examples}
  \small
\begin{lstlisting}
prompt = "Enter a number: "
while True:
    try:
        x = int(input(prompt))
        break
    except ValueError:
        print("Invalid input, try again...")

\end{lstlisting}
\end{frame}

\begin{frame}[fragile]
  \frametitle{Catching multiple exceptions}
  \small
\begin{lstlisting}
while True:
    try:
        data = input()
        x = int(data.split(',')[1])
        break
    except IndexError:
        print('Input at least 2 values.')
    except ValueError:
        print("Invalid input, try again...")

\end{lstlisting}
\end{frame}


\begin{frame}[fragile]
  \frametitle{Catching multiple exceptions}
\begin{lstlisting}
data = input()
try:
    x = int(data.split(',')[1])
except (ValueError, IndexError):
    print("Invalid input ...")

\end{lstlisting}
\end{frame}

\begin{frame}[fragile]
  \frametitle{\typ{try, except, else}}
  \small
\begin{lstlisting}
while True:
    try:
        data = input()
        x = int(data.split(',')[1])
    except (ValueError, IndexError):
        print("Invalid input ...")
    else:
        print('All is well!')
        break
  \end{lstlisting}

\end{frame}


\begin{frame}
  \frametitle{Some comments}
  \begin{itemize}
  \item In practice NEVER use blanket except clauses
  \item Always catch specific exceptions
  \end{itemize}
\end{frame}

\begin{frame}[fragile]
  \frametitle{Exceptions: raising your exceptions}
\small
\begin{lstlisting}
>>> raise ValueError("your error message")
Traceback (most recent call last):
  File "<stdin>", line 2, in ?
ValueError: your error message
\end{lstlisting}
\end{frame}



\begin{frame}[fragile]
  \frametitle{Exceptions: try/finally}
  \small
\begin{lstlisting}
while True:
    try:
        x = int(input(prompt))
        break
    except ValueError:
        print("Invalid number, try again...")
    finally:
        print("All good!")

      \end{lstlisting}
      \normalsize
      Always runs the finally clause!
\end{frame}


\begin{frame}[fragile]
  \frametitle{Exceptions: try/finally}
  \begin{lstlisting}
def f(x):
    try:
        y = int(x)
        return y
    except ValueError:
        print(x)
    finally:
        print('finally')

>>> f(1)
>>> f('a')
\end{lstlisting}
Always runs the finally clause!
\end{frame}


\begin{frame}
  \frametitle{Summary}
  \begin{itemize}
  \item Catching exceptions with \typ{try/except}
  \item Catching multiple exceptions
  \item Cleanup with \typ{finally}
  \item Raising your own exceptions
  \end{itemize}
\end{frame}

\begin{frame}
  \frametitle{What next?}
  \begin{itemize}
  \item Only covered the very basics
  \item More advanced topics remain
  \item Read the official Python tutorial:
    \url{docs.python.org/tutorial/}
  \end{itemize}
\end{frame}

\end{document}
