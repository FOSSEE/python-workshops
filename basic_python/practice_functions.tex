%%%%%%%%%%%%%%%%%%%%%%%%%%%%%%%%%%%%%%%%%%%%%%%%%%%%%%%%%%%%%%%%%%%%%%%%%%%%%%%%
% Tutorial slides on Python.
%
% Author: FOSSEE
% Copyright (c) 2017, FOSSEE, IIT Bombay
%%%%%%%%%%%%%%%%%%%%%%%%%%%%%%%%%%%%%%%%%%%%%%%%%%%%%%%%%%%%%%%%%%%%%%%%%%%%%%%%

\documentclass[14pt,compress]{beamer}


% Modified from: generic-ornate-15min-45min.de.tex
\mode<presentation>
{
  \usetheme{Madrid}
  \useoutertheme{infolines}
  \setbeamercovered{invisible}
}

\usepackage[english]{babel}
\usepackage[latin1]{inputenc}
%\usepackage{times}
\usepackage[T1]{fontenc}

% Taken from Fernando's slides.
\usepackage{ae,aecompl}
\usepackage{mathpazo,courier,euler}
\usepackage[scaled=.95]{helvet}

\definecolor{darkgreen}{rgb}{0,0.5,0}

\usepackage{listings}
\lstset{language=Python,
    basicstyle=\ttfamily\bfseries,
    commentstyle=\color{red}\itshape,
  stringstyle=\color{darkgreen},
  showstringspaces=false,
  keywordstyle=\color{blue}\bfseries}

%%%%%%%%%%%%%%%%%%%%%%%%%%%%%%%%%%%%%%%%%%%%%%%%%%%%%%%%%%%%%%%%%%%%%%
% Macros
\setbeamercolor{emphbar}{bg=blue!20, fg=black}
\newcommand{\emphbar}[1]
{\begin{beamercolorbox}[rounded=true]{emphbar}
      {#1}
 \end{beamercolorbox}
}
\newcounter{time}
\setcounter{time}{0}
\newcommand{\inctime}[1]{\addtocounter{time}{#1}{\tiny \thetime\ m}}

\newcommand{\typ}[1]{\textbf{\texttt{{#1}}}}


\newcommand{\kwrd}[1]{ \texttt{\textbf{\color{blue}{#1}}}  }

%%% This is from Fernando's setup.
% \usepackage{color}
% \definecolor{orange}{cmyk}{0,0.4,0.8,0.2}
% % Use and configure listings package for nicely formatted code
% \usepackage{listings}
% \lstset{
%    language=Python,
%    basicstyle=\small\ttfamily,
%    commentstyle=\ttfamily\color{blue},
%    stringstyle=\ttfamily\color{orange},
%    showstringspaces=false,
%    breaklines=true,
%    postbreak = \space\dots
% }

%\pgfdeclareimage[height=0.75cm]{iitmlogo}{iitmlogo}
%\logo{\pgfuseimage{iitmlogo}}


%% Delete this, if you do not want the table of contents to pop up at
%% the beginning of each subsection:
\AtBeginSubsection[]
{
  \begin{frame}<beamer>
    \frametitle{Outline}
    \tableofcontents[currentsection,currentsubsection]
  \end{frame}
}

\AtBeginSection[]
{
  \begin{frame}<beamer>
    \frametitle{Outline}
    \tableofcontents[currentsection,currentsubsection]
  \end{frame}
}

% If you wish to uncover everything in a step-wise fashion, uncomment
% the following command:
%\beamerdefaultoverlayspecification{<+->}

%\includeonlyframes{current,current1,current2,current3,current4,current5,current6}


%%%%%%%%%%%%%%%%%%%%%%%%%%%%%%%%%%%%%%%%%%%%%%%%%%%%%%%%%%%%%%%%%%%%%%
% Title page
\title[Basic Python]{Practice exercises: functions}

\author[FOSSEE Team] {The FOSSEE Group}

\institute[FOSSEE -- IITB] {Department of Aerospace Engineering\\IIT Bombay}
\date[] {Mumbai, India}
%%%%%%%%%%%%%%%%%%%%%%%%%%%%%%%%%%%%%%%%%%%%%%%%%%%%%%%%%%%%%%%%%%%%%%


%%%%%%%%%%%%%%%%%%%%%%%%%%%%%%%%%%%%%%%%%%%%%%%%%%%%%%%%%%%%%%%%%%%%%%
% DOCUMENT STARTS
\begin{document}

\begin{frame}
  \titlepage
\end{frame}

\begin{frame}[fragile,plain]
  \frametitle{Note: Python 2.x and 3.x}

 If you are using Python 2.x
  \begin{itemize}
  \item Use \typ{raw\_input} instead of \typ{input}
  \item Use the following for \typ{print}
  \end{itemize}
 \begin{lstlisting}
from __future__ import print_function
\end{lstlisting}
\end{frame}

\begin{frame}[plain]
  \frametitle{Exercise: function without arguments}
  \begin{enumerate}
  \item Define a function called \typ{prompt}
  \item The function should ask the user to enter their name (no prompt)
  \item The function should not take any arguments
  \item The function should not return anything
  \item The function should print \typ{Hello <name>}
  \end{enumerate}
  For example if the user enters \typ{Sam}, print:
  Hello Sam
\end{frame}

\begin{frame}[fragile,plain]
\frametitle{Solution}
\begin{lstlisting}
  def prompt():
      name = input()
      print('Hello', name)
\end{lstlisting}
\end{frame}

\begin{frame}[plain]
  \frametitle{Exercise: function with one argument}
  \begin{enumerate}
  \item Define a function called \typ{prompt}
  \item The function should take a single string argument
  \item \textbf{Do not ask the user for input, i.e. do not use input}
  \item The function should not return anything
  \item The function should print \typ{Hello <name>}
  \end{enumerate}
  For example if the function is passed \typ{'Sam'}, print:
  Hello Sam
\end{frame}

\begin{frame}[fragile,plain]
\frametitle{Solution}
\begin{lstlisting}
  def prompt(name):
      print('Hello', name)
\end{lstlisting}
\end{frame}


\begin{frame}[plain,fragile]
  \frametitle{Exercise: function with \typ{return}}
  \begin{enumerate}
  \item Define a function called \typ{prompt}
  \item The function should take a single string argument
  \item The function should return a string with \typ{'Hello'}
  \item \textbf{Do not use input}
  \item \textbf{Do not print anything}
  \end{enumerate}
  For example if the function is passed \typ{'Sam'}, return:
  \begin{lstlisting}
    'Hello Sam'
  \end{lstlisting}
\end{frame}

\begin{frame}[fragile,plain]
\frametitle{Solution}
\begin{lstlisting}
  def prompt(name):
      return 'Hello ' + name
\end{lstlisting}
\end{frame}

\begin{frame}[plain]
  \frametitle{Exercise: function with two args}
  \begin{enumerate}
  \item Define a function called \typ{add}
  \item The function should take two arguments
  \item The function should return the sum of the two arguments
  \item \textbf{Do not use input}
  \item \textbf{Do not print anything}
  \end{enumerate}
\end{frame}

\begin{frame}[fragile,plain]
\frametitle{Solution}
\begin{lstlisting}
  def add(a, b):
      return a + b
\end{lstlisting}
\end{frame}

\begin{frame}[plain]
  \frametitle{Exercise: function returning boolean}
  \begin{enumerate}
  \item Define a function called \typ{is\_even}
  \item The function should take a single integer argument
  \item The function should return \typ{True} if the number is even and \typ{False} otherwise
  \item \textbf{Do not use input}
  \item \textbf{Do not print anything}
  \end{enumerate}
\end{frame}

\begin{frame}[fragile,plain]
\frametitle{Naive Solution}
\begin{lstlisting}
  def is_even(x):
      if x%2 == 0:
          return True
      else:
          return False
\end{lstlisting}
\end{frame}

\begin{frame}[fragile,plain]
\frametitle{Elegant Solution}
\begin{lstlisting}
  def is_even(x):
      return x%2 == 0
\end{lstlisting}
\end{frame}

\begin{frame}[plain,fragile]
  \frametitle{Exercise: function returning two values}
  \begin{enumerate}
  \item Define a function called \typ{even\_square}
  \item The function should take a single argument
  \item The function should return if the number is even and the square of the number
  \item \textbf{Do not use input}
  \item \textbf{Do not print anything}
  \end{enumerate}
  For example:
  \begin{lstlisting}
    In []: even_square(2)
    Out[]: (True, 4)
    In []: even_square(3)
    Out[]: (False, 9)
  \end{lstlisting}
\end{frame}

\begin{frame}[fragile,plain]
  \frametitle{Solution}
  \begin{lstlisting}
  def even_square(x):
      return x%2 == 0, x*x
    \end{lstlisting}
\end{frame}


\begin{frame}[plain,fragile]
  \frametitle{Exercise: default arguments}
  \begin{enumerate}
  \item Define a function called \typ{greet}
  \item The function should take one positional argument, \typ{name}
  \item The function should take one optional argument, \typ{message}
  \item If \typ{message} is not given, it should default to \typ{'Hello'}
  \item It should return the string with the greeting
  \end{enumerate}
  For example:
  \begin{lstlisting}
    In []: greet('Sam')
    Out[]: 'Hello Sam'
    In []: greet('Sam', 'Hi')
    Out[]: 'Hi Sam'
  \end{lstlisting}
\end{frame}

\begin{frame}[fragile,plain]
  \frametitle{Solution}
  \begin{lstlisting}
    def greet(name, message='Hello'):
        return message + ' ' + name
  \end{lstlisting}
\end{frame}

\begin{frame}[plain, fragile]
  \frametitle{Exercise: functions and lists}
  \begin{enumerate}
  \item Define a function called \typ{to\_lower}
  \item The function should take a single list of strings
  \item The function should return the list of strings but all in lowercase
  \item \textbf{Do not use input}
  \item \textbf{Do not print anything}
  \end{enumerate}
  For example:
  \begin{lstlisting}
In []: to_lower(['I', 'am', 'Batman'])
Out[]: ['i', 'am', 'batman']
  \end{lstlisting}
\end{frame}

\begin{frame}[fragile,plain]
\frametitle{Solution}
\begin{lstlisting}
  def to_lower(data):
      result = []
      for x in data:
          result.append(x.lower())
      return result
\end{lstlisting}
\end{frame}

\begin{frame}[plain,fragile]
  \frametitle{Exercise: list of Fibonacci}
  \begin{enumerate}
  \item Define a function called \typ{fib} taking one argument \typ{n}
  \item Where, \typ{n>0} is an integer but defaults to 8
  \item Return the first \typ{n} terms of the Fibonacci sequence
  \end{enumerate}
  For example:
  \begin{lstlisting}
    In []: fib(4)
    Out[]: [0, 1, 1, 2]
    In []: fib()
    Out[]: [0, 1, 1, 2, 3, 5, 8, 13]
  \end{lstlisting}
\end{frame}

\begin{frame}[fragile,plain]
\frametitle{Solution}
\begin{lstlisting}
  def fib(n=8):
      a, b = 0, 1
      result = [0]
      for i in range(n-1):
          result.append(b)
          a, b = b, a+b
      return result
\end{lstlisting}
\end{frame}

\begin{frame}[plain,fragile]
  \frametitle{Exercise: returning a function}
  \begin{enumerate}
  \item Define a function called \typ{power2()} which takes no argument
  \item It should return a function which takes a single argument \typ{x} but
    returns $2^x$
  \end{enumerate}
  For example:
  \begin{lstlisting}
    In []: f = power2()
    In []: f(2)
    Out[]: 4
    In []: power2()(4)
    Out[]: 16
  \end{lstlisting}
\end{frame}

\begin{frame}[fragile,plain]
\frametitle{Solution}
\begin{lstlisting}
  def power2():
      def f(x):
          return 2**x
      return f
\end{lstlisting}
\end{frame}

\begin{frame}[fragile,plain]
  \frametitle{Another solution}
  \begin{lstlisting}
    def power(n=2):
        def f(x):
            return n**x
        return f
   \end{lstlisting}
   \begin{itemize}
   \item This is called a closure.
   \item Note that \typ{f} ``stores'' the value of \typ{n}
   \end{itemize}
   \pause
   \begin{lstlisting}
     In []: p2 = power(2)
     In []: p3 = power(3)
     In []: p2(2)
     Out[]: 4
     In []: p3(2)
     Out[]: 9
  \end{lstlisting}
\end{frame}

\begin{frame}[plain,fragile]
  \frametitle{Exercise: function as an argument}
  \begin{enumerate}
  \item Define a function called \typ{apply(f, data)}
  \item Where \typ{f} is a function taking a single value
  \item Where \typ{data} is a list
  \item It should return a list where the function is applied to each element
    of \typ{data}
  \end{enumerate}
  For example:
  \begin{lstlisting}
    In []: def double(x):
    .....:    return 2*x
    .....:
    In []: apply(double, [1, 2, 3])
    Out[]: [2, 4, 6]
  \end{lstlisting}
\end{frame}

\begin{frame}[plain, fragile]
  \frametitle{Solution}
  \begin{lstlisting}
    def apply(f, data):
        result = []
        for x in data:
            result.append(f(x))
        return result
  \end{lstlisting}
\end{frame}

\begin{frame}
  \centering
  \Huge

  That's all folks!
\end{frame}
\end{document}

\end{document}
